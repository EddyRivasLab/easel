The easel (esl) module implements a small set of functionality shared
by all the modules: notably, the error-handling system.

\subsection{Error handling conventions}

Easel distinguishes two types of error conditions: ``normal'' errors
that are ``returned'', and ``abnormal'' errors that are ``thrown''.

Normal errors are conditions that are expected, that an application
should want to know about. They often have to do with issues with user
input; an EOF end-of-file indicator for an input file, for example. An
Easel function that detects a normal error directly returns an
appropriate normal status code (for example, by calling \ccode{return
ESL\_EOF}.

Abnormal errors are conditions that the application wishes would never
happen. Most common is a lack of system resources (principally, memory
allocation failures). An Easel function that detects an abnormal error
condition does two things: 1) it calls the \cfunc{esl\_error()} error
handler; and 2) if \cfunc{esl\_error} did not terminate the program,
it returns an abnormal status code.

By default, \cfunc{esl\_error()} just prints a short error message and
aborts execution. Therefore, by default, Easel functions deal with
their own problems (drastically) and never actually return these
abnormal error codes. Thus, simple applications can rely on Easel to
handle its own errors -- provided you don't mind that any fatal error
in Easel will kill your whole program.

In complex applications (graphical user interfaces, for example) where
robustness is at a premium, and you really don't want a little failure
in a low-level library crashing your whole environment, the default
behavior of \cfunc{esl\_error} can be modified by assigning a new
error handler that merely catches the information from Easel and
reacts appropriately, instead of the default behavior of
\ccode{abort()}'ing.

By convention, the documentation of individual Easel functions refers
to abnormal error conditions that are handled by \cfunc{esl\_error}
before returning an error status code as \emph{thrown}, and normal
error conditions that directly return an error status code as
\emph{returned}.  From the perspective of an application developer
using Easel, any returned error codes must always be checked for. If
\cfunc{esl\_error} is left in its default, thrown errors do not need
to be checked for, because \cfunc{esl\_error} handles them itself (by
aborting). In a complex application where a custom nonfatal handler
has been assigned to \cfunc{esl\_error()}, the application must check
for both returned (normal) and thrown (abnormal) error codes.

\subsection{Status codes}

For most functions, the return status code is an integer,
\cmacro{ESL\_OK} (0) on success and nonzero on failure. The error
codes are defined in \cfile{easel.h}:

\begin{cchunk}
#define ESL_OK         0	/* no error                     */
#define ESL_EOL        1	/* end-of-line (often normal)   */
#define ESL_EOF        2	/* end-of-file (often normal)   */
#define ESL_EMEM       3	/* malloc or realloc failed     */
#define ESL_ENOFILE    4	/* file not found               */
#define ESL_EFORMAT    5	/* file format not recognized   */
#define ESL_EPARAM     6	/* bad parameter passed to func */
#define ESL_EDIVZERO   7	/* attempted div by zero        */
#define ESL_EINCOMPAT  8	/* incompatible parameters      */
#define ESL_EINVAL     9	/* invalid argument             */
#define ESL_ETESTFAIL  10	/* calculated test failure      */
#define ESL_EUNKNOWN   127      /* generic error, unidentified  */
\end{cchunk}

Functions that allocate memory may instead return a valid pointer or
NULL on failure. Functions that do this have no possible errors other
than malloc() failure, so that a NULL return is synonymous with
\cmacro{ESL\_EMEM}.

Some functions have no detectable error conditions. These functions
may return a meaningful result instead of a status code, or they may
just return void. Easel is not dogmatic about requiring every function
to return a status code.

Whether a code is ``normal'' or ``abnormal'' depends on the function.
The documentation for an individual function will say what normal
status codes are returned directly, and what abnormal status codes can
be thrown after calling \cfunc{esl\_error}.

\subsection{Replacing the default error handler}

\cfunc{esl\_error} gets as arguments a status code, file and
linenumber for where the error occurred, and a \cfunc{printf}-style
message. By default, it prints the message and exits:

\begin{cchunk}
Easel fatal error:
Memory allocation failed.

Aborted at file sqio.c, line 42. 
\end{cchunk}

An application can define its own handler for the same information,
and override this default behavior. To do this, define the error
handler with the following prototype:

\begin{cchunk}
extern void my_error_handler(int code, char *file, int line, char *format, va_list arg);
\end{cchunk}

An example implementation of an error handler:

\begin{cchunk}
#include <stdarg.h>

void
my_error_handler(int code, char *file, int line, char *format, va_list arg)
{
  fprintf(stderr, ``Easel threw an error (code %d):\n'', code);
  vfprintf(stderr, format, arg);
  fprintf(stderr, ``at line %d, file %s\b'', line, file);
  return;
}
\end{cchunk}

To configure Easel to use your error handler, call
\cfunc{esl\_error\_SetHandler(\&my\_error\_handler)}. Usually you
would do this before calling any other Easel functions.

In principle, you can change error handlers at any time, including
restoring the default handler with
\cfunc{esl\_error\_RestoreDefaultHandler()}. However, the
implementation of the handler uses a static function pointer that is
not threadsafe, so in a threaded program, you would need to make sure
that multiple threads do not try to change the handler at the same
time.

Since Easel functions also call Easel functions, the function that
detected an error may not be the function that your application
called.  If you implement a handler that does not exit the program, an
abnormal error in Easel will generate a whole stack trace of
\cfunc{esl\_error} messages, as the abnormal error percolates up from
the function that detected the error, until the Easel function you
called throws the abnormal error code back to your application. The
first \cfunc{esl\_error} message is most relevant; any remaining
messages arise from that error percolating up through the stack trace.
A sophisticated replacement \cfunc{esl\_error} handler might push each
\cfunc{esl\_error} message into a FIFO queue, where they will be
waiting for the main application-specific error handler to access when
the application gets its abnormal status code back from Easel.

\subsection{Implementation of Easel's error-throwing convention}

Easel implements its two-step error throwing convention (a call to
\cfunc{esl\_error} with \cmacro{\_\_FILE\_\_} and \cmacro{\_\_LINE\_\_}
information, followed by a return of a status code) in two macros,
\cmacro{ESL\_ERROR(code, mesg)} (for integer error codes) and
\cmacro{ESL\_ERROR\_NULL(mesg)} (for memory allocations that return
NULL). An example:

\begin{cchunk}
if ((ptr = malloc(sizeof(int) * n)) == NULL) ESL_ERROR_NULL(``malloc failed'');
\end{cchunk}

Vanilla ANSI C does not support macros with variable arguments, so
these macros can only be used when a message does not need to be
formatted. When the message must be formatted, the two-step convention
is called explicitly, as in this example:

\begin{cchunk}
if ((ptr = malloc(sizeof(int) * n)) == NULL)
  {
     esl_error(ESL_EMEM, __FILE__, __LINE__, ``malloc of %d ints failed'', n);
     return NULL;
  }
\end{cchunk}

For memory allocation and reallocation, Easel may use two macros
\cmacro{ESL\_MALLOC} and \cmacro{ESL\_REALLOC}, which encapsulate
standard \cfunc{malloc} and \cfunc{realloc} calls in Easel's
error-throwing convention.


