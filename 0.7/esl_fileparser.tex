
The fileparser module parses an input data file into one token at a
time, allowing for arbitrary comments and whitespace.

\subsection{The fileparser API}

The module implements one object, a \ccode{ESL\_FILEPARSER}. This
object holds the state of the parser.

The API contains the following functions:

\vspace{1em}
\begin{center}
\begin{tabular}{ll}\hline
\ccode{esl\_fileparser\_Create()}        & Creates fileparser, given input stream.\\
\ccode{esl\_fileparser\_SetCommentChar()}& Declares what character starts a comment.\\
\ccode{esl\_fileparser\_GetToken()}      & Parse the next token.\\
\ccode{esl\_fileparser\_Destroy()}       & Free's a created \ccode{ESL\_FILEPARSER}.\\\hline
\end{tabular}
\end{center}

\subsection{Example of using the fileparser API}

An example that opens a file, reads all its tokens one at a time, and
prints out token number, token length, and the token itself:

\input{cexcerpts/fileparser_example}

The caller opens an input \ccode{FILE} stream normally (using
\ccode{fopen()}, for example, to open a file). This input stream is
associated with a new \ccode{ESL\_FILEPARSER} by calling
\ccode{esl\_fileparser\_Create}.

A single character can be defined to serve as a comment character
(often \ccode{\#}), using the \ccode{esl\_fileparser\_SetCommentChar()}
call. The parser will ignore the comment character, and the remainder
of any line following a comment character.

Each call to \ccode{esl\_fileparser\_GetToken()} retrieves one
whitespace-delimited token from the input stream; the call returns
\ccode{eslOK} if a token is parsed, and \ccode{eslEOF} when there are
no more tokens in the file. Whitespace is defined as space, tab,
newline, or carriage return (\verb+" \t\n\r"+).

If the token is not what you expected (or if
\ccode{esl\_fileparser\_GetToken()} fails for some reason, with a
status code other than \ccode{<eslOK>} or \ccode{<eslEOF>}), you
probably want to provide some diagnostic output to the user. For this
reason, \ccode{efp->linenumber} contains the current linenumber in the
file.  The actual contents of this line are not available in the
\ccode{efp} object, because the input line buffer (\ccode{efp->buf})
is modified as successive tokens are parsed out.

When the caller is done, the fileparser is free'd by a call to
\ccode{esl\_fileparser\_Destroy()}. The caller is also responsible for
closing the stream in the appropriate manner (for instance by
\ccode{fclose()}, if it was opened by \ccode{fopen()}).



