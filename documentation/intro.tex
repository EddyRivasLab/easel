\subsection{Overview of modules}
\begin{tabular}{lll}\hline
\textbf{Module}            & \textbf{Description}                       & \textbf{Requires}\\\hline
  \multicolumn{3}{c}{\textbf{Core modules}}\\
easel                      & Framework for using Easel                  &  - \\
alphabet                   & Digitized biosequence alphabets            & easel\\
dmatrix                    & Matrix algebra                             & easel\\ 
getopts                    & Command line parsing \& application config & easel\\
msa                        & Multiple sequence alignment i/o            & easel\\
parse                      & File and line parsing routines             & easel\\
random                     & Random number generator                    & easel\\
regexp                     & Regular expression matching                & easel\\
stack                      & Pushdown stacks                            & easel\\
vectorops                  & Vector operations                          & easel\\\hline
  \multicolumn{3}{c}{\textbf{Derived modules}}\\
sqio                       & Sequence file i/o                          & easel alphabet\\
bioparse\_paml             & PAML rate matrix datafiles                 & easel dmatrix parse \\
ratematrix                 & Evolutionary rate matrices                 & easel dmatrix vectorops\\
gamma                      & Gamma densities and Gamma function         & easel random\\
dirichlet                  & Dirichlet densities                        & easel gamma random\\ \hline     
  \multicolumn{3}{c}{\textbf{Optional library interfaces}}\\
interface\_gsl             & GNU Scientific Library                     & easel dmatrix\\
interface\_lapack          & LAPACK linear algebra library              & easel dmatrix \\\hline
\end{tabular}



\subsection{Overview of data structures}

\begin{tabular}{lll}\hline
\textbf{Object}          & \textbf{Implemented in} & \textbf{Description}\\\hline
\ccode{ESL\_ALPHABET}    & \cfile{alphabet}        & Digitized sequence alphabet\\
\ccode{ESL\_DMATRIX}     & \cfile{dmatrix}         & 2D double-precision matrix for linear algebra \\
\ccode{ESL\_FILEPARSER}  & \cfile{parse}           & Simple token-based input file parser\\
\ccode{ESL\_GETOPTS}     & \cfile{getopts}         & Application configuration state\\
\ccode{ESL\_MSA}         & \cfile{msa}             & Multiple sequence alignment\\
\ccode{ESL\_MSAFILE}     & \cfile{msa}             & Multiple sequence alignment file parser\\
\ccode{ESL\_PERMUTATION} & \cfile{dmatrix}         & Permutation matrix used in linear algebra\\
\ccode{ESL\_RANDOMNESS}  & \cfile{random}          & Random number generator\\
\ccode{ESL\_REGEXP}      & \cfile{regexp}          & Regular expression pattern-matching machine\\
\ccode{ESL\_SEQFILE}     & \cfile{sqio}            & Biosequence file parser (unaligned)\\
\ccode{ESL\_SQ}          & \cfile{sqio}            & DNA/RNA/protein sequence data\\
\ccode{ESL\_STACK}       & \cfile{stack}           & Pushdown stack\\\hline
\end{tabular}

\subsection{Function naming conventions in Easel}

\subsubsection{Object creation, initialization, destruction}

Most of Easel's objects are allocated on the heap; that is, accessed
exclusively via pointers. Less often, routines may allow an object to
be allocated on the stack.

\begin{sreitems}{\ccode{Create,Destroy}}
\item [\ccode{Create,Destroy}] 
  \ccode{esl\_foo\_Create()} allocates and initializes a new \ccode{ESL\_FOO}
  object, returning a pointer to the new
  object. The \ccode{Create()} function is passed any necessary
  initialization or size information in its arguments.
  \ccode{esl\_foo\_Destroy(obj)} frees all the memory associated
  with a \ccode{ESL\_FOO} object,  given a pointer to the object,
  and returns \ccode{ESL\_OK}. Example:

\begin{cchunk}
ESL_SQ *sq;

sq = esl_sq_Create();
esl_sq_Destroy(sq);
\end{cchunk}
  
\item [\ccode{Open,Close}] 
  Same as \ccode{Create()} and \ccode{Destroy()}, but specifically for
  objects associated with input/output streams. Example:

\begin{cchunk}
char        *seqfile = ``foo.seq'';
ESL_SIO     *sqfp;

sqfp = esl_sio_Open(seqfile);
esl_sio_Close(sqfp);
\end{cchunk}


\item [\ccode{Inflate,Deflate}]
  \ccode{esl\_foo\_Inflate(\&obj)} takes a pointer to the ``shell'' of an
  \ccode{ESL\_FOO} object that has been allocated on the stack, and
  allocates and initializes all the internals of it. The
  \ccode{Inflate()} function is passed any necessary initialization or
  size information in its arguments.  \ccode{esl\_foo\_Deflate(\&obj)}
  frees all the internal memory associated with a \ccode{ESL\_FOO} object,
  given a pointer to the object, but the object shell is left alone.
  The only difference between \ccode{Create,Destroy} and
  \ccode{Inflate,Deflate} is whether the object shell itself is to be
  allocated, or not. Example:

\begin{cchunk}
ESL_SQ  sq;

esl_sq_Inflate(&sq);
esl_sq_Deflate(&sq);
\end{cchunk}



\item [\ccode{Expand,Squeeze}]
   These deal with objects whose contents are not necessarily of a
   fixed size, but can grow and require reallocation of internal data
   fields. A function \ccode{esl\_foo\_Expand()} reallocates an
   \ccode{ESL\_FOO} object to a larger size. Usually, reallocation
   works by doubling the previous allocation. The redoubling strategy
   can result in a fair amount of unused memory overhead (up to
   50\%). A function \ccode{esl\_foo\_Squeeze()} takes a fully-grown
   object and optimizes its memory usage, recovering this wastage
   overhead, on the assumption that no more reallocation will be
   done.



\item [\ccode{Reuse}] 
   \ccode{esl\_foo\_Reuse(obj)} reinitializes an object exactly as a
   \ccode{Create} or \ccode{Inflate} function would initialize it, \emph{without}
   allocating new memory; it reuses memory that has
   already been allocated when the object was originally created or
   inflated. For some objects that are used sequentially (like,
   sequences), reusing one object saves malloc()'s compared to
   lots of Create/Destroy calls. A \ccode{Reuse} function does not
   care whether the object was originally created by a \ccode{Create}
   or a \ccode{Inflate} call. Example:

\begin{cchunk}
ESL_SQ *sq;

sq = esl_sq_Create();
  /* read a sequence into the sq object, have fun with it */

esl_sq_Reuse(sq);
  /* read a second sequence into it */

esl_sq_Destroy(sq);
\end{cchunk}

\end{sreitems}

\subsubsection{Other common object manipulation functions}

\begin{sreitems}{\ccode{\_Copy(src, dest)}}

\item[\ccode{\_Copy(src, dest)}]
Copies \ccode{src} object into \ccode{dest}, where the caller has
already created the empty \ccode{dest} object. Returns \ccode{ESL\_OK}
on success; throws \ccode{ESL\_EINCOMPAT} if the objects are not
compatible (for example, two matrices that are not the same size).

The order of the arguments is always \ccode{src} $\rightarrow$
\ccode{dest} (unlike the C library's \ccode{strcpy()} convention, which
is the opposite order).

\item[\ccode{\_Duplicate(obj)}] 

Creates and returns a pointer to a duplicate of \ccode{obj}.
Equivalent to (and is a shortcut for) \ccode{dest = \_Create();
\_Copy(src, dest)}. Caller is responsible for free'ing the duplicate
object, just as if it had been \ccode{\_Create}'d. Throws NULL if
allocation fails.

\item[\ccode{\_Set*(obj, value...)}]

Initializes value(s) in \ccode{obj} to \ccode{value}. Special cases of
\ccode{\_Set*} functions may exist, like \ccode{\_SetZero} (set to
zero(s)), or \ccode{esl\_dmatrix\_SetIdentity} (set a dmatrix to an
identity matrix).

\end{sreitems}


\subsection{Discipline of writing Easel modules}


\subsubsection{API}

\begin{enumerate}
\item Exposed function names obey Easel conventions.

\item All \ccode{ret\_*} pointers for retrieving info from 
      a function are implemented as optional.

\item On any error (returned or thrown), a function releases any
      memory it has allocated, and all \ccode{ret\_*} pointers are
      \ccode{NULL} or 0, depending on their type.

\item Any function that calls another Easel function must 
      catch any thrown errors. Within Easel, we
      can't assume that the error handler is set to be a fatal
      one.
\end{enumerate}


\subsubsection{Documentation}

\begin{enumerate}
\item Documentation in .tex files is written to application developers
      (and me); documentation in .c comments is written to Easel
      developers (and me).

\item Every module has a .tex file documenting the module and its API,
      and (if it makes sense) also the particular implementation.

\item Every function exposed to the API can be autodocumented (comment
      header can be converted to \LaTeX) with
      \ccode{autodoc\_functions}, in order to produce appendices
      summarizing the complete Easel function set.
\end{enumerate}


\subsubsection{Testing}

\begin{enumerate}

\item Every module has a test driver which exercises the API in some
      simple way (and, in the process, provides a ``hello world''
      level example of using the module).

\end{enumerate}

\begin{quote}
Lack of skill dictates economy of style.\\
 -- Joey Ramone
\end{quote}     





