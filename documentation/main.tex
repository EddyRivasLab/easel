\documentclass[11pt]{article}
\usepackage{fullpage}
\usepackage{times}
\usepackage{html}
\usepackage{apalike}
\usepackage{url}
\usepackage{fancyvrb}

% Typography.
\newcommand{\ccode}[1]{{\small\texttt{#1}}}
\newcommand{\emcode}[1]{{\small\bfseries\texttt{#1}}}
\newcommand{\prog}[1]{\small\texttt{#1}}
\newcommand{\eslmod}[1]{{\smaller\bfseries\texttt{#1}}}
\newcommand{\eslfunc}[1]{\hyperlink{man:#1}{{\small\texttt{#1}}}}

% Eliminate the ones below someday. 
\newcommand{\cmacro}[1]{\texttt{#1}}
\newcommand{\cvar}[1]{\texttt{#1}}
\newcommand{\cvartype}[1]{\texttt{#1}}
\newcommand{\cstruct}[1]{\texttt{#1}}
\newcommand{\cfunc}[1]{\texttt{#1}}
\newcommand{\cfile}[1]{\texttt{#1}}

\def\argmax{\mathop{\mathrm{argmax}}\limits}
\def\argmin{\mathop{\mathrm{argmin}}\limits}

\DefineVerbatimEnvironment{cchunk}{Verbatim}{fontsize=\scriptsize,xleftmargin=2.0\parindent}%

% Description-like environment for documenting functions/APIs.
% puts the description label in a minipage with a large hanging
% indent.
% Good christ this took a long time to develop.
% hanging indent trick stolen from Peter Wilson's hanging.sty @CTAN
% minipage allows multi-line label, and puts item on next line.
% customized list inspired by Kopka/Daly _Guide to LaTeX_ p.213
% SRE, Wed Dec 27 11:37:18 2000
%
\newenvironment{sreapi}{%
     \begin{list}{}{%
       \renewcommand{\makelabel}[1]{%
         \begin{minipage}{\textwidth}%
           \hangindent10em\hangafter1\noindent%
           {\bfseries\texttt{##1}\vspace{0.8em}}%
         \end{minipage}%
     }}}%
     {\end{list}}


% Description-like environment for producing lists like:
%
%     label  stuff, stuff, stuff
%
%    label2  more stuff, more stuff,
%            more stuff.
% \begin{sreitems}{Longest label} \item[label] stuff, ... \end{sreitems}
% SRE, Wed Dec 27 11:59:43 2000
%
\newenvironment{sreitems}[1]{%
     \begin{list}{}{%
       \settowidth{\labelwidth}{#1}%
       \setlength{\leftmargin}{\labelwidth}%
       \addtolength{\leftmargin}{\labelsep}%
       }}
     {\end{list}}


\begin{document}
\bibliographystyle{apalike}

\begin{titlepage}
{\Large

\vspace*{\fill}

\noindent
{\Huge{Easel}} \\ 
\rule[2pt]{\textwidth}{1pt} \\
\hspace*{\fill} {\large {A library of C functions for
    biological sequence analysis} \\ }

\vspace*{\fill}

\begin{center}
\url{http://selab.janelia.org/easel/}\\
Version 0.1; March 2008 \\ 

\vspace*{\fill}

Sean R. Eddy\\
HHMI Janelia Farm Research Campus\\
19700 Helix Drive\\
Ashburn VA 20147\\
\url{http://selab.janelia.org/}\\
\end{center}

\vspace*{\fill}
}
\end{titlepage}

\vspace*{\fill}
\begin{flushleft}
Copyright (C) 2004-2007 Sean R. Eddy.\\
Copyright (C) 2006-2007 HHMI Janelia Farm Research Campus.\\
Copyright (C) 2004-2006 Washington University School of Medicine.\\
\end{flushleft}



\newpage
\tableofcontents
\newpage

\section{Easel}


\subsection{Overview of data structures}

\begin{tabular}{lll}\hline
\textbf{Object}            & \textbf{Implemented in} & \textbf{Description}\\\hline
\cstruct{ESL\_DMATRIX}     & \cfile{dmatrix}         & 2D double-precision matrix for linear algebra \\
\cstruct{ESL\_PERMUTATION} & \cfile{dmatrix}         & Permutation matrix used in linear algebra\\
\cstruct{ESL\_FILEPARSER}  & \cfile{parse}           & Simple token-based input file parser\\
\cstruct{ESL\_RANDOMNESS}  & \cfile{random}          & Random number generator\\
\cstruct{ESL\_SEQFILE}     & \cfile{sqio}            & Biosequence file parser\\
\cstruct{ESL\_SQ}          & \cfile{sqio}            & DNA/RNA/protein sequence data\\\hline
\end{tabular}

\section{Error-handling conventions}

\begin{quote}
\emph{
Only a complete outsider could ask your question. Are there control
authorities? There are nothing but control authorities. Of course,
their purpose is not to uncover errors in the ordinary meaning of the
word, since errors do not occur and even when an error does in fact
occur, as in your case, who can say conclusively that it is an error?\\
\hspace*{\fill} - Franz Kafka, \emph{The Castle} }
\end{quote}


Easel distinguishes two types of error conditions: ``normal'' and
``abnormal''.

Normal errors are conditions that are expected, and that an
application wants to know about. They often have to do with user input
streams; an EOF end-of-file indicator for an input file, for
example. An Easel function that detects a normal error directly
returns an appropriate normal status code.

Abnormal errors are conditions that the application wishes would never
happen. Most common is a lack of system resources (principally, memory
allocation failures). An Easel function that detects an abnormal error
condition does two things: 1) it calls the \cfunc{esl\_error()} error
handler; and 2) if \cfunc{esl\_error} did not terminate the program,
it returns an abnormal status code.

By default, \cfunc{esl\_error()} just prints a short error message and
aborts execution. Therefore, by default, Easel functions deal with
their own problems (drastically) and do not return abnormal error
codes. Thus, simple applications can rely on Easel to handle its own
errors, provided you don't mind that any fatal error in Easel will
kill your whole program.

In complex applications (graphical user interfaces, for example) where
robustness is at a premium, and you really don't want a little failure
in a low-level library crashing your whole environment, the default
behavior of \cfunc{esl\_error} can be modified by assigning a new
error handler that merely catches the information from Easel.

By convention, the documentation of individual Easel functions refers
to abnormal error conditions that are handled by \cfunc{esl\_error}
before returning an error status code as \emph{thrown}, and normal
error conditions that directly return an error status code as
\emph{returned}.  From the perspective of an application developer
using Easel, any returned error codes must always be checked for. If
\cfunc{esl\_error} is left in its default, thrown errors do not need
to be checked for, because \cfunc{esl\_error} will handle them itself
(by aborting). In a complex application where a custom nonfatal
handler has been assigned to \cfunc{esl\_error()}, the application
must check for both returned (normal) and thrown (abnormal) error
codes.

\subsection{Status codes}

For most functions, the return status code is an integer,
\cmacro{ESL\_OK} (0) on success and nonzero on failure. The error
codes are defined in \cfile{easel.h}:

\begin{cchunk}
#define ESL_OK         0	/* no error                     */
#define ESL_EOL        1	/* end-of-line (often normal)   */
#define ESL_EOF        2	/* end-of-file (often normal)   */
#define ESL_EMEM       3	/* malloc or realloc failed     */
#define ESL_ENOFILE    4	/* file not found               */
#define ESL_EFORMAT    5	/* file format not recognized   */
#define ESL_EPARAM     6	/* bad parameter passed to func */
#define ESL_EDIVZERO   7	/* attempted div by zero        */
#define ESL_EINCOMPAT  8	/* incompatible parameters      */
#define ESL_EINVAL     9	/* invalid argument             */
#define ESL_ETESTFAIL  10	/* calculated test failure      */
#define ESL_EUNKNOWN   127      /* generic error, unidentified  */
\end{cchunk}

Functions that allocate memory may instead return a valid pointer or
NULL on failure. Functions that do this have no possible errors other
than malloc() failure, so that a NULL return is synonymous with
\cmacro{ESL\_EMEM}.

Some functions have no detectable error conditions. These functions
may return a meaningful result instead of a status code. Easel is not
dogmatic about requiring every function to return a status code.

Whether a code is ``normal'' or ``abnormal'' depends on the function.
The documentation for an individual function will say what normal
status codes are returned directly, and what abnormal status codes can
be thrown after calling \cfunc{esl\_error}.

\subsection{Replacing the default error handler}

\cfunc{esl\_error} gets as arguments a status code, file and
linenumber for where the error occurred, and a \cfunc{printf}-style
message. By default, it prints the message and exits:

\begin{cchunk}
Easel fatal error:
Memory allocation failed.

Aborted at file sqio.c, line 42. 
\end{cchunk}

An application can define its own handler for the same information,
and override this default behavior. To do this, define the error
handler with the following prototype:

\begin{cchunk}
extern void my_error_handler(int code, char *file, int line, char *format, va_list arg);
\end{cchunk}

An example implementation of an error handler:

\begin{cchunk}
#include <stdarg.h>

void
my_error_handler(int code, char *file, int line, char *format, va_list arg)
{
  fprintf(stderr, ``Easel threw an error (code %d):\n'', code);
  vfprintf(stderr, format, arg);
  fprintf(stderr, ``at line %d, file %s\b'', line, file);
  return;
}
\end{cchunk}

To configure Easel to use your error handler, call
\cfunc{esl\_error\_SetHandler(\&my\_error\_handler)}. Usually you would do
this before calling any other Easel functions.

In principle, you can change error handlers at any time, including
restoring the default handler with
\cfunc{esl\_error\_RestoreDefaultHandler()}. However, the implementation
of the handler uses a static function pointer that is not threadsafe,
so in a threaded program, you would need to make sure that multiple
threads do not try to change the handler at the same time.

Since Easel functions also call Easel functions, the function that
detected an error may not be the function that you called.  If you
implement a handler that does not exit the program, an abnormal error
in Easel will generate a whole stack trace of \cfunc{esl\_error}
messages, as the abnormal error percolates up from the function that
detected the error, until the Easel function you called throws the
abnormal error code back to your application. The first
\cfunc{esl\_error} message is most relevant; any remaining messages
arise from that error percolating up through the stack trace.  A
sophisticated replacement \cfunc{esl\_error} handler might push each
\cfunc{esl\_error} message into a FIFO queue, where they will be
waiting for the main application-specific error handler to access when
the application gets its abnormal status code back from Easel.

\subsection{Implementation of Easel's error-throwing convention}

Easel implements its two-step error throwing convention (a call to
\cfunc{esl\_error} with \cmacro{\_\_FILE\_\_} and \cmacro{\_\_LINE\_\_}
information, followed by a return of a status code) in two macros,
\cmacro{ESL\_ERROR(code, mesg)} (for integer error codes) and
\cmacro{ESL\_ERROR\_NULL(mesg)} (for memory allocations that return
NULL). An example:

\begin{cchunk}
if ((ptr = malloc(sizeof(int) * n)) == NULL) ESL_ERROR_NULL(``malloc failed'');
\end{cchunk}

Vanilla ANSI C does not support macros with variable arguments, so
these macros can only be used when a message does not need to be
formatted. When the message must be formatted, the two-step convention
is called explicitly, as in this example:

\begin{cchunk}
if ((ptr = malloc(sizeof(int) * n)) == NULL)
  {
     esl_error(ESL_EMEM, __FILE__, __LINE__, ``malloc of %d ints failed'', n);
     return NULL;
  }
\end{cchunk}






\section{Function naming conventions}

\subsection{Object creation, initialization, destruction}

Most of Easel's objects are allocated on the heap; that is, accessed
exclusively via pointers. Less often, routines may allow an object to
be allocated on the stack.

\begin{sreitems}{\cfunc{Create,Destroy}}
\item [\cfunc{Create,Destroy}] 
  \cfunc{esl\_foo\_Create()} allocates and initializes a new \cstruct{ESL\_FOO}
  object, returning a pointer to the new
  object. The \cfunc{Create()} function is passed any necessary
  initialization or size information in its arguments.
  \cfunc{esl\_foo\_Destroy(obj)} frees all the memory associated
  with a \cstruct{ESL\_FOO} object,  given a pointer to the object.
  Example:

\begin{cchunk}
ESL_SQ *sq;

sq = esl_sq_Create();
esl_sq_Destroy(sq);
\end{cchunk}
  
\item [\cfunc{Open,Close}] 
  Same as \cfunc{Create()} and \cfunc{Destroy()}, but specifically for
  objects associated with input/output streams. Example:

\begin{cchunk}
char        *seqfile = ``foo.seq'';
ESL_SEQFILE *sqfp;

sqfp = esl_seqfile_Open(seqfile);
esl_seqfile_Close(sqfp);
\end{cchunk}

\item [\cfunc{Init,Release}]
  \cfunc{esl\_foo\_Init(\&obj)} takes a pointer to the ``shell'' of an
  \cstruct{ESL\_FOO} object that has been allocated on the stack, and
  allocates and initializes all the internals of it. The
  \cfunc{Init()} function is also passed any necessary initialization or
  size information in its arguments.  \cfunc{esl\_foo\_Release(\&obj)}
  frees all the internal memory associated with a \cstruct{ESL\_FOO} object,
  given a pointer to the object, but the object shell is left alone.
  The only difference between \cfunc{Create,Destroy} and
  \cfunc{Init,Release} is whether the object shell itself is to be
  allocated, or not. Example:

\begin{cchunk}
ESL_SQ  sq;

esl_sq_Init(&sq);
esl_sq_Release(&sq);
\end{cchunk}


\item [\cfunc{Reuse}] 
   \cfunc{esl\_foo\_Reuse(obj)} reinitializes an object (exactly as a
   \cfunc{Create} or \cfunc{Init} function would initialize it), \emph{without}
   allocating new memory; it reuses memory that has
   already been allocated when the object was originally created or
   initialized. For some objects that are used sequentially (like,
   sequences), reusing one object saves malloc()'s compared to
   lots of Create/Destroy calls. A \cfunc{Reuse} function does not
   care whether the object was originally created by a \cfunc{Create}
   or a \cfunc{Init} call. Example:

\begin{cchunk}
ESL_SQ *sq;

sq = esl_sq_Create();
  /* read a sequence into the sq object, have fun with it */
esl_sq_Reuse(sq);
  /* read a second sequence into it */
esl_sq_Destroy(sq);
\end{cchunk}



\end{sreitems}


\end{document}

