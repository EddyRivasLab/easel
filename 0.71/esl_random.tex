The \eslmod{random} module contains routines for generating
pseudorandom numbers. The heart of the module is the
\ccode{esl\_random()} pseudorandom number generator. Other elementary
sampling functions provided by the random module rely on
\ccode{esl\_random()} as their random number source.

\ccode{esl\_random()} is portable, and it gives reproducible results
on all platforms. In contrast, ANSI C \ccode{rand()} implementations
(for example) in system libraries will typically differ on different
platforms.

\ccode{esl\_random()} is reentrant and threadsafe. Many generator
implementations are not, including implementations of ANSI C
\ccode{rand()}.

The \ccode{esl\_random()} algorithm is a strong one. It is essentially
the \ccode{ran2()} generator from \emph{Numerical Recipes in C}
\citep{Press93}, implementing L'Ecuyer's algorithm for combining two
linear congruential generators, with a Bays-Durham shuffle. It
generates pseudorandom double-precision real numbers on the interval
$0.0 \leq x < 1.0$, with a minimum spacing of about 4.7e-10, by
normalizing a random integer on the interval 0..2147483562.  According
to Press \emph{et al.}, it has a period of $> 2 \times
10^{18}$. 

The bit stream generated internally by \ccode{esl\_random()} passes 14
tests in a National Institute of Standards and Technology statistical
benchmark for cryptanalysis \citep{Rukhin01}, with performance
comparable to random binary sequences from irrational
numbers.\footnote{The NIST tests are necessary but not sufficient to
show adequate ``randomness''. Indeed, though the bitstream does pass
NIST benchmarks, the 31 individual bits in \ccode{esl\_random()}'s
internal unnormalized random integer are not random, because its
maximum value is slightly less than $2^{31}-1$. Only the double value
returned by \ccode{esl\_random()} should be used as a random sample,
or the integer value of \ccode{r->rnd}, but not the 31 individual bits
of \ccode{r->rnd}.  The fact that the bitstream passes the NIST tests
just indicates that there are no glaring problems in the generator.}

Table~\ref{tbl:random_api} lists the functions in the \eslmod{random}
API. The module implements one object, \ccode{ESL\_RANDOMNESS}, which
contains state information for the random number generator.  This
makes random number generation reentrant and threadsafe. You can have
more than one active generator and they will not interfere with each
other. The object is meant to be opaque; you should not need to use
its contents.  (One possible exception: if you want to use the random
integer $0..2147483562$ instead of the normalized double-precision
real that \ccode{esl\_random()} returned, you can access
\ccode{r->rnd} in the \ccode{ESL\_RANDOMNESS} object \ccode{r} after a
\ccode{esl\_random(r)} call.)

\begin{table}[bhp]
\begin{center}
\begin{tabular}{ll}\hline
   \multicolumn{2}{c}{\textbf{the \ccode{ESL\_RANDOMNESS} object}}\\
\ccode{esl\_randomness\_Create()}           & Creates new random number generator.\\
\ccode{esl\_randomness\_CreateTimeSeeded()} & Creates new generator using current time as seed.\\
\ccode{esl\_randomness\_Destroy()}          & Free's a random number generator.\\
\ccode{esl\_randomness\_Init()}             & Re-seeds a random number generator.\\
   \multicolumn{2}{c}{\textbf{random number sampling}}\\
\ccode{esl\_random()}                       & Returns uniform deviate $0.0 \leq x < 1.0$.\\
\ccode{esl\_rnd\_UniformPositive()}         & Returns uniform deviate $0.0 < x < 1.0$.\\
\ccode{esl\_rnd\_Gaussian()}                & Gaussian-distributed deviate.\\
\ccode{esl\_rnd\_Gamma()}                   & $\Gamma(a)$-distributed deviate.\\
\ccode{esl\_rnd\_Choose()}                  & Choose a uniformly-distributed random integer.\\
\ccode{esl\_rnd\_DChoose()}                 & Choose integer according to a probability vector.\\
\ccode{esl\_rnd\_FChoose()}                 & Choose integer according to a probability vector.\\
\ccode{esl\_rnd\_IID()}                     & Generate an i.i.d. symbol sequence.\\ \hline
\end{tabular}
\end{center}
\caption{The \eslmod{random} API.}
\label{tbl:random_api}
\end{table}


\subsection{Example of using the random API}

Figure~\ref{fig:random_example} shows a program that initializes the
random number generator with a seed of ``42'', then samples 10 random
numbers using \ccode{esl\_random()}.

\begin{figure}
\input{cexcerpts/random_example}
\caption{An example of using the random number generator.}
\label{fig:random_example}
\end{figure}

When a \ccode{ESL\_RANDOMNESS} object is created with
\ccode{esl\_randomness\_Create()}, it needs to be given a \emph{seed},
an integer $> 0$, which specifies the initial state of the
generator. After a generator is seeded, it is typically never seeded
again. A series of \ccode{esl\_random()} calls generates a
pseudorandom number sequence from that starting point. If you create
two \ccode{ESL\_RANDOMNESS} objects seeded identically, they are
guaranteed to generate the same random number sequence on all
platforms. This makes it possible to reproduce stochastic simulations.
Thus, if you run the example multiple times, you get the same ten
numbers, because the generator is always seeded with 42.

Often one wants different runs to generate different random number
sequences, which creates a chicken and the egg problem: how can we
select a pseudorandom seed for the pseudorandom number generator? One
standard method is to use the current time. Easel provides an
alternate creation function
\ccode{esl\_randomness\_CreateTimeseeded()}. If the precision of our
clock is sufficiently fine, different runs of the program occur at
different clock times. Easel relies on the POSIX clock for
portability, but the POSIX clock has the drawback that it clicks in
seconds, which is not very good precision. Two different
\ccode{ESL\_RANDOMNESS} objects created in the same second will
generate identical random number sequences. Change the
\ccode{esl\_randomness\_Create(42)} call to
\ccode{esl\_randomness\_CreateTimeseeded()} and recompile, and you'll
get different number sequences with each run - provided you run the
commands in different seconds, anyway.

