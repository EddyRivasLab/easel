
The \eslmod{keyhash} (key) module contains routines for emulating Perl
associative arrays, by associating keywords with an integer array
index, and storing the association in a hash table for rapid access.

\subsection{The keyhash API}

The module implements one object: the hash table, an
\ccode{ESL\_KEYHASH}.

The API defines five functions: 

\begin{tabular}{ll}
\ccode{esl\_keyhash\_Create()}  & Create a new \ccode{ESL\_KEYHASH}. \\
\ccode{esl\_keyhash\_Destroy()} & Destroys a created \ccode{ESL\_KEYHASH}. \\
\ccode{esl\_keyhash\_Dump()}    & Dumps internal info from a \ccode{ESL\_KEYHASH}. \\
\ccode{esl\_key\_Store()}       & Stores a key, gets an associated index.\\
\ccode{esl\_key\_Lookup()}      & Looks up a key, returns index or -1.\\
\end{tabular}

\subsection{Example of using the keyhash API}

The idea behind using the keyhash module is shown in this fragment of
pseudocode:

\begin{cchunk}
       #include <easel.h>
       #include <esl_keyhash.h>
     
       ESL_KEYHASH *hash;
       int   idx;
       char  key;
       
       hash = esl_keyhash_Create();
 (Storing:) 
       (foreach key) {
          esl_key_Store(hash, key, &idx);       
          (reallocate foo, bar as needed)
          foo[idx] = whatever;
          bar[idx] = whatever;
       }     
 (Reading:)
       (foreach key) {
          if (esl_key_Lookup(hash, key, &idx) != eslOK) { no_such_key; }
          (do something with) foo[idx];
          (do something with) bar[idx];
       }   
       esl_keyhash_Destroy();
\end{cchunk}

That is, the application maintains info in normal C-style arrays that
are indexed by an integer index value, and it uses the keyhash to
associate a specific keyword with an integer index. To store info, you
first store the keyword and obtain a new index (this simply starts at
0 and counts up, as you store successive keys), then you store the
info in appropriate arrays at that index. To look up info, you look up
the keyword to obtain the index, then you access the info by indexing
into your arrays.

This is a (laborious) moral equivalent of Perl's associative arrays, as
in \ccode{\$foo\{\$key\} = whatever; \$bar\{\$key\} = whatever}.

The following (contrived) example is real C code to store keywords
listed from one file, and looks up keywords listed in a second
file. It doesn't demonstrate the idea of using the index to store and
retrieve additional info associated with the keyword, but it does
demonstrate the real C API to keyhash.

\begin{cchunk}
#include <stdio.h>
#include <easel/easel.h>
#include <easel/keyhash.h>

int
main(int argc, char **argv)
{
  FILE        *fp;
  char         buf[256];
  char        *s, *tok;
  ESL_KEYHASH *h;
  int          idx;
  int          nstored, nsearched, nshared;

  h = esl_keyhash_Create();

  /* Read/store keys from file 1.
   */
  if ((fp = fopen(argv[1], "r")) == NULL)
    { fprintf(stderr, "couldn't open %s\n", argv[1]); exit(1); }
  nstored = 0;
  while (fgets(buf, 256, fp) != NULL)
    {
      s = buf;
      esl_strtok(&s, " \t\r\n", &tok, NULL);
      esl_key_Store(h, tok, &idx);
      nstored++;
    }
  fclose(fp);
  printf("Stored %d keys.\n", nstored);

  /* Look up keys from file 2.
   */
  if ((fp = fopen(argv[2], "r")) == NULL)
    { fprintf(stderr, "couldn't open %s\n", argv[2]); exit(1); }
  nsearched = nshared = 0;
  while (fgets(buf, 256, fp) != NULL)
    {
      s = buf;
      esl_strtok(&s, " \t\r\n", &tok, NULL);

      if (esl_key_Lookup(h, tok, &idx) == eslOK) nshared++;
      nsearched++;
    }
  fclose(fp);
  printf("Looked up %d keys.\n", nsearched);
  printf("In common: %d keys.\n", nshared);

  esl_keyhash_Destroy(h);
  return 0;
}
\end{cchunk}
