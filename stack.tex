The stack module implements pushdown stacks for storing integers,
characters, or arbitrary pointers (objects).

The module follows a convention of prepending \ccode{I}, \ccode{C},
\ccode{P} to \ccode{Create()}, \ccode{Push()}, and \ccode{Pop()}
function names, to indicate the stack's datatype as integer,
character, or pointer, respectively. For example,
\ccode{esl\_stack\_PCreate()} creates a stack for pointer
storage. (This is also the naming convention in the \ccode{vectorops}
module.)  Stacks can be thought of as typed objects, with the type
defined by the \ccode{Create()} call; types may not be mixed for any
particular created stack. (That is, you may not create a character
stack and try to push an integer onto it.)

\subsection{API}
   
An example of using the integer stack, pushing 42, 7, and 3 on, then
popping them off:

\begin{cchunk}
#include <easel/easel.h>
#include <easel/stack.h>

int main(void)
{
   ESL_STACK *ns;
   int        x;

   ns = esl_stack_ICreate();
   esl_stack_IPush(ns, 42);
   esl_stack_IPush(ns, 7);
   esl_stack_IPush(ns, 3);
   while (esl_stack_IPop(ns, &x) != eslEOD) 
      printf("%d\n", x);
   esl_stack_Destroy(ns);   
}
\end{cchunk}

The \ccode{Create()} functions create a stack for a particular
purpose. \ccode{esl\_stack\_ICreate()} creates a stack for integers,
\ccode{esl\_stack\_CCreate()} creates a stack for characters, and
\ccode{esl\_stack\_PCreate()} creates a stack for pointers.  They
throw NULL if an allocation fails.  All three stack types are free'd
by a call to \ccode{esl\_stack\_Destroy()}. All three types can also
be reused without reallocation or recreation by
\ccode{esl\_stack\_Reuse()}. A \ccode{Reuse()}'d stack retains its
original datatype.

The \ccode{Push()} functions push one datum onto the stack, of the
appropriate type. They throw \ccode{eslEMEM} if the stack needs to
reallocate internally but fails.

The \ccode{Pop()} functions pop one datum off the stack, returning it
through a passed pointer. They return \ccode{eslOK} on success, and
\ccode{eslEOD} if the stack is empty.

\ccode{esl\_stack\_ObjectCount()} returns the number of objects stored
in the stack. 

A special function, \ccode{esl\_stack\_Convert2String()}, operates
only on character stacks. It converts the stack structure to a
NUL-terminated string, with characters in the same order they were
pushed. The stack is destroyed by this operation, leaving a
\ccode{char *} behind.

\subsection{Allocation strategy}

Stacks are initially allocated for a certain number of objects,
defined by a compile-time constant \ccode{ESL\_STACK\_INITALLOC} in
\ccode{stack.h}. The default is 128. Whenever a stack needs to grow,
it reallocates by doubling its current allocation.


