The \eslmod{msa} module implements input parsers and output routines
for multiple sequence alignment files. 

The module implements two objects. A \ccode{ESL\_MSA} holds a multiple
sequence alignment. A \ccode{ESL\_MSAFILE} is an alignment file,
opened for input. (No special object is needed for output of an
alignment file. A normal C \ccode{FILE} stream is used for output.)
The API for the \eslmod{msa} module is summarized in
Table~\ref{tbl:msa_api}.

MSAs are normally stored in an \ccode{ESL\_MSA} as text data.
Augmentation with the \eslmod{alphabet} module allows MSAs to be input
as digital data, or converted to and from digital data. 

Augmentation with the \eslmod{ssi} module allows rapid random access
of SSI-indexed large MSA database files like Pfam or Rfam. When
augmented, the \ccode{esl\_msafile\_Open()} and
\ccode{esl\_msafile\_OpenDigital()} functions automatically open an
accompanying SSI index, if it is present.

Augmentation with the \eslmod{keyhash} module allows much better
performance on parsing large Stockholm alignment files, by
accelerating indexing of some internal data structures.

If Easel is compiled with optional MPI library support enabled,
routines for communicating \ccode{ESL\_MSA} objects via MPI messages
are available.

\begin{table}[hbp]
\begin{center}
{\small
\begin{tabular}{|ll|}\hline
\apisubhead{The ESL\_MSA object                                           }\\
\hyperlink{func:esl_msa_Create()}{\ccode{esl\_msa\_Create()}} & Creates an \ccode{ESL\_MSA} object.\\
\hyperlink{func:esl_msa_CreateFromString()}{\ccode{esl\_msa\_CreateFromString()}} & Creates a small \ccode{ESL\_MSA} from a test case string.\\
\hyperlink{func:esl_msa_Destroy()}{\ccode{esl\_msa\_Destroy()}} & Frees an \ccode{ESL\_MSA}.\\
\hyperlink{func:esl_msa_Expand()}{\ccode{esl\_msa\_Expand()}} & Reallocate for more sequences.\\
\apisubhead{The ESL\_MSAFILE object                                       }\\
\hyperlink{func:esl_msafile_Open()}{\ccode{esl\_msafile\_Open()}} & Open an MSA file for input.\\
\hyperlink{func:esl_msafile_Close()}{\ccode{esl\_msafile\_Close()}} & Closes an open MSA file.\\
\hyperlink{func:esl_msafile_PositionByKey()}{\ccode{esl\_msafile\_PositionByKey()}} & Use SSI to reposition file to start of named MSA.\\
\apisubhead{Digitized MSA's (alphabet augmentation required)}\\
\hyperlink{func:esl_msa_GuessAlphabet()}{\ccode{esl\_msa\_GuessAlphabet()}} & Guess alphabet of MSA.\\
\hyperlink{func:esl_msa_CreateDigital()}{\ccode{esl\_msa\_CreateDigital()}} & Create a digital \ccode{ESL\_MSA}.\\
\hyperlink{func:esl_msa_Digitize()}{\ccode{esl\_msa\_Digitize()}} & Digitizes an msa, converting it from text mode.\\
\hyperlink{func:esl_msa_Textize()}{\ccode{esl\_msa\_Textize()}} & Convert a digital msa to text mode.\\
\hyperlink{func:esl_msafile_GuessAlphabet()}{\ccode{esl\_msafile\_GuessAlphabet()}} & Guess what kind of sequences the alignment file contains.\\
\hyperlink{func:esl_msafile_OpenDigital()}{\ccode{esl\_msafile\_OpenDigital()}} & Open an msa file for digital input.\\
\hyperlink{func:esl_msafile_SetDigital()}{\ccode{esl\_msafile\_SetDigital()}} & Set an open \ccode{ESL\_MSAFILE} to read in digital mode.\\
\apisubhead{MPI communication. (alphabet augmentation and MPI required)}\\
\hyperlink{func:esl_msa_MPISend()}{\ccode{esl\_msa\_MPISend()}} & Send essential msa info to MPI destination.\\
\hyperlink{func:esl_msa_MPIRecv()}{\ccode{esl\_msa\_MPIRecv()}} & Receive essential MSA info from an MPI sender.\\
\apisubhead{General i/o API, all alignment formats                                 }\\
\hyperlink{func:esl_msa_Read()}{\ccode{esl\_msa\_Read()}} & Read next MSA from a file.\\
\hyperlink{func:esl_msa_Write()}{\ccode{esl\_msa\_Write()}} & Write an MSA to a file.\\
\hyperlink{func:esl_msa_DescribeFormat()}{\ccode{esl\_msa\_DescribeFormat()}} & Convert internal file format code to text string.\\
\hyperlink{func:esl_msa_GuessFileFormat()}{\ccode{esl\_msa\_GuessFileFormat()}} & Determine the format of an open MSA file.\\
\apisubhead{Miscellaneous functions for manipulating MSAs}\\
\hyperlink{func:esl_msa_SequenceSubset()}{\ccode{esl\_msa\_SequenceSubset()}} & Select subset of sequences into a smaller MSA.\\
\hyperlink{func:esl_msa_MinimGaps()}{\ccode{esl\_msa\_MinimGaps()}} & Remove columns containing all gym symbols.\\
\hyperlink{func:esl_msa_NoGaps()}{\ccode{esl\_msa\_NoGaps()}} & Remove columns containing any gap symbol.\\
\hyperlink{func:esl_msa_SymConvert()}{\ccode{esl\_msa\_SymConvert()}} & Global search/replace of symbols in an MSA.\\
\hline
\end{tabular}
}
\end{center}
\caption{The \eslmod{msa} API.}
\label{tbl:msa_api}
\end{table}

\subsection{Example of using the msa API}

To read an alignment (or alignments) from a file, you open the file
with \ccode{esl\_msafile\_Open()} and read the alignment(s) with
\ccode{esl\_msa\_Read()}. When you're done with an alignment, you free
it with \ccode{esl\_msa\_Destroy()}. When you're done with the file,
you close it with \ccode{esl\_msafile\_Close()}.

To output an alignment, open a normal C \ccode{FILE} stream, write the
alignment(s) with \ccode{esl\_msa\_Write()}, and close the stream with
C's \ccode{fclose()}.

You may not need the other functions in the object and i/o API, unless
you are writing your own routine for creating
MSAs. \ccode{esl\_msa\_Read()} calls the \ccode{esl\_msa\_Create()}
and \ccode{esl\_msa\_Expand()} functions as part of its
work. \ccode{esl\_msafile\_Open()} calls
\ccode{esl\_msa\_GuessFileFormat()} in order to do format
autodetection.

Anyway, on to an example. The code bloats up a bit here because of
error handling. Lots of things can go wrong with input from a user's
file format, so the API tries to be careful about detecting different
kinds of errors. Here's code that reads one or more alignment in from
a file, and outputs them in Stockholm format:

\input{cexcerpts/msa_example}

Some things that are special about the API are worth noting.

\begin{enumerate}
\item The format of the alignment file can either be automatically
      detected, or set by the caller when the file is opened.
      Autodetection is invoked when the caller passes a \ccode{fmt}
      code of \ccode{eslMSAFILE\_UNKNOWN}. Autodetection is a ``best
      effort'' guess, but it is not 100\% reliable - especially if the
      input file isn't an alignment file at all. Autodetection is a
      convenient default but the caller will usually want to provide a
      way for the user to specify the input file format and override
      autodetection, just in case.

\item If reading of an alignment fails because something is wrong with
      the file format, it's useful to give the user more information
      about what went wrong than just ``parse failed''. If
      \ccode{esl\_msa\_Read()} returns an \ccode{eslEFORMAT} error
      when trying to read an alignment from an open file \ccode{afp},
      the caller can use \ccode{afp->linenumber}, \ccode{afp->buf},
      and \ccode{afp->errbuf} to get the line number in the file that
      the error occurred, the text that was on that line, and a short
      error message about what was wrong with it, respectively.

\item Note the example of \eslmod{keyhash} augmentation, just by
      including \ccode{esl\_keyhash.h} header. The effects of
      \eslmod{keyhash} augmentation are all internal the \eslmod{msa}
      module, rather than providing any new functions.
\end{enumerate}

\subsection{Accessing alignment data}

The information in the \ccode{ESL\_MSA} object is meant to be accessed
directly, so you need to know what it contains. This object is defined
and documented in \ccode{esl\_msa.h}. It contains various information,
as follows:

\subsubsection{Important/mandatory information}

The following information is always available in an MSA (except
digital-mode alignments, which replace \ccode{aseq[][]} with
\ccode{ax[][]}, as described later):

\input{cexcerpts/msa_mandatory}

The alignment contains \ccode{nseq} sequences, each of which contains
\ccode{alen} characters.

\ccode{aseq[i]} is the i'th aligned sequence, numbered
\ccode{0..nseq-1}. \ccode{aseq[i][j]} is the j'th character in aligned
sequence i, numbered \ccode{0..alen-1}.

\ccode{sqname[i]} is the name of the i'th sequence.

\ccode{wgt[i]} is a non-negative real-valued weight for sequence
i. This defaults to 1.0 if the alignment file did not provide weight
data. You can determine whether weight data was parsed by checking
\ccode{flags \& MSA\_HASWGTS}.

\subsubsection{Optional information}

The following information is optional; it is usually only provided by
annotated Stockholm alignments (for instance, Pfam and Rfam database
alignments). 

Any pointer can be NULL if the information is unavailable. This is
true at any level; for instance, \ccode{ss} will be NULL if no
secondary structures are available for any sequence, and \ccode{ss[i]}
will be NULL if some secondary structures are available, but not for
sequence i. 

\input{cexcerpts/msa_optional}

These should be self-explanatory; but for more information, see the
Stockholm format documentation. Each of these fields corresponds to
Stockholm markup.

The \ccode{cutoff} array contains Pfam/Rfam curated score
cutoffs. They are indexed as follows:

\input{cexcerpts/msa_cutoffs}

\subsubsection{Unparsed information}

The MSA object may also contain additional ``unparsed'' information
from Stockholm files; that is, tags that are present but not
recognized by the MSA module. This information is stored so that it
may be regurgitated if the application needs to faithfully output the
entire alignment file, even the bits that it didn't understand. If you
need to access unparsed Stockholm tags, see the comments in
\ccode{esl\_msa.h}.

\subsubsection{Off-by-one issues in indexing alignment columns}

With one exception, all arrays over alignment columns are normal C
string arrays, indexed \ccode{0..alen-1}. This includes optional
information such as \ccode{msa->rf[]} (the reference annotation line)
and \ccode{msa->cs[]} (the consensus structure annotation line).

The exception is a digitized sequence alignment, \ccode{msa->ax[][]}
(see below), where columns are indexed 1..alen and sentinel bytes at
positions 0 and alen+1, following Easel's convention for digitized
sequences.

Thus, when your code is manipulating a digitized alignment and using
optional information like the reference annotation line or the
consensus structure line, you must be careful of the off-by-one
difference in how the two types of data are indexed.

\subsection{Accepted formats}

Currently, the MSA module only parses Stockholm format. 

Stockholm format and other alignment formats are documented in a later
chapter.

\subsection{Digital versus text representation}

The multiple alignment is normally stored as ASCII text symbols in a
2D array \ccode{char ** aseq[0..nseq-1][0..alen-1]}. Optionally, when
augmented with the \eslmod{alphabet} module, the multiple alignment
may alternatively be stored as digital data in an Easel internal
alphabet.

An \ccode{ESL\_MSA} may therefore be in either \esldef{text mode} or
\esldef{digital mode}. Text mode is the default behavior. An
\ccode{ESL\_MSA} is in digital mode if its \ccode{eslMSA\_DIGITAL} flag
is up (\ccode{msa->flags \& eslMSA\_DIGITAL} is \ccode{TRUE}). When the
alignment data are in digital mode, they are stored internally as a 2D
digital sequence array \ccode{ESL\_DSQ ** ax[0..nseq-1][1..alen]}, and
the \ccode{aseq} field is \ccode{NULL}.

To use a digital internal representation, it is most efficient to read
directly as digital data, using a \ccode{esl\_msafile\_OpenDigital()}
call in place of \ccode{esl\_msafile\_Open()}. You can also change the
mode of an MSA from text to digital using
\ccode{esl\_msa\_Digitize()}, and digital to text using
\ccode{esl\_msa\_Textize()}.

\subsection{Reading from stdin or gzip-compressed files}

The module can read compressed alignment files.  If the
\ccode{filename} passed to \ccode{esl\_msafile\_Open()} ends in
\ccode{.gz}, the file is assumed to be compressed with gzip. Instead
of opening it normally, \ccode{esl\_msafile\_Open()} opens it as a pipe
through \ccode{gunzip -dc}. Obviously this only works on a POSIX
system -- pipes have to work, specifically the \ccode{popen()} system
call -- and \ccode{gunzip} must be installed and in the PATH.

The module can also read from a standard input pipe. If the
\ccode{filename} passed to \ccode{esl\_msafile\_Open()} is \ccode{-},
the alignment is read from \ccode{STDIN} rather than from a file.

Because of the way format autodetection works, you cannot use it when
reading from a pipe or compressed file. The application must know the
appropriate format and pass that code it calls
\ccode{esl\_msafile\_Open()}.
