The random (rnd) module contains routines for generating pseudorandom
numbers.

The standard ANSI C library provides a random number generator in
\ccode{rand()}. However, the implementation of \ccode{rand()} varies
from platform to platform, thwarting reproducibility of results that
depend on being able to reproduce the same pseudorandom number series.
Moreover, on many platforms \ccode{rand()} is weak. It is desirable to
have a standard, portable, strong pseudorandom number generator that
generates reproducible pseudorandom number series on any platform.
Thus Easel's \ccode{esl\_random()} function, which returns a
double-precision uniform deviate $x$ on the interval $0 \leq x < 1$.

\subsection{The random API}

The module implements one object, \ccode{ESL\_RANDOMNESS}, which
contains state information for the random number generator.  This
object is meant to be opaque; you should not need to use its contents.
Because the state of the generator is encapsulated this way, random
number generation is reentrant and threadsafe. We can have more than
one active generator and they will not interfere with each other. 

The API consists of the following functions:

\vspace{1em}
\begin{tabular}{ll}\hline
   \multicolumn{2}{c}{\textbf{the \ccode{ESL\_RANDOMNESS} object}}\\
\ccode{esl\_randomness\_Create()}           & Creates new random number generator.\\
\ccode{esl\_randomness\_CreateTimeSeeded()} & Creates new generator using current time as seed.\\
\ccode{esl\_randomness\_Destroy()}          & Free's a random number generator.\\
\ccode{esl\_randomness\_Init()}             & Re-seeds a random number generator.\\
   \multicolumn{2}{c}{\textbf{random number sampling}}\\
\ccode{esl\_random()}                       & Returns uniform deviate $0.0 \leq x < 1.0$.\\
\ccode{esl\_rnd\_UniformPositive()}         & Returns uniform deviate $0.0 < x < 1.0$.\\
\ccode{esl\_rnd\_Exponential()}             & Exponentially distributed deviate $0 < x \leq \infty$.\\
\ccode{esl\_rnd\_Gaussian()}                & Gaussian-distributed deviate.\\
\ccode{esl\_rnd\_Choose()}                  & Choose a uniformly-distributed random integer.\\
\ccode{esl\_rnd\_DChoose()}                 & Choose integer according to a probability vector.\\
\ccode{esl\_rnd\_FChoose()}                 & Choose integer according to a probability vector.\\
\ccode{esl\_rnd\_IID()}                     & Generate an i.i.d. symbol sequence.\\ \hline
\end{tabular}

The algorithm implemented by \ccode{esl\_random()} is strong, but a
bit slow compared to some other strong random number generators. It is
essentially the \ccode{ran2()} generator from \emph{Numerical Recipes
in C} \citep{Press93}. It implements L'Ecuyer's algorithm for
combining two linear congruential generators, with a Bays-Durham
shuffle. The other elementary sampling functions provided by the
random module all rely on the \ccode{esl\_random()} generator for
their random numbers.


\subsection{Example of using the random API}

An example that initializes the random number generator with a seed of
``42'', then samples 10 random numbers using \ccode{esl\_random()}:

\input{cexcerpts/random_example}

When a \ccode{ESL\_RANDOMNESS} object is created with
\ccode{esl\_randomness\_Create()}, it needs to be given a \emph{seed},
an integer $> 0$, which specifies the initial state of the
generator. After a generator is seeded, it is typically never seeded
again. A series of \ccode{esl\_random()} calls generates a
pseudorandom number sequence from that starting point. If you create
two \ccode{ESL\_RANDOMNESS} objects seeded identically, they are
guaranteed to generate the same random number sequence on all
platforms. This makes it possible to reproduce stochastic simulations.
Thus, if you run the example multiple times, you get the same ten
numbers, because the generator is always seeded with 42.

Often one wants different runs to generate different random number
sequences, which creates a chicken and the egg problem: how can we
select a pseudorandom seed for the pseudorandom number generator? One
standard method is to use the current time. Easel provides an
alternate creation function
\ccode{esl\_randomness\_CreateTimeseeded()}. If the precision of our
clock is sufficiently fine, different runs of the program occur at
different clock times. Easel relies on the POSIX clock for
portability, but the POSIX clock has the drawback that it clicks in
seconds, which is not very good precision. Two different
\ccode{ESL\_RANDOMNESS} objects created in the same second will
generate identical random number sequences. Change the
\ccode{esl\_randomness\_Create(42)} call to
\ccode{esl\_randomness\_CreateTimeseeded()} and recompile, and you'll
get different number sequences with each run - provided you run the
commands in different seconds, anyway.





