The \eslmod{sq} module provides \Easel's object for single biological
sequences.

Sequences may either be in text mode or digital mode. In text mode,
sequences are in ASCII text, in a normal C string. In digital mode,
sequences are predigested into Easel's internal format for fast
biosequence handling. The idea of providing both modes is that text
mode might be simpler to use (you don't have to know anything about
Easel's digital alphabet system), whereas digital mode is more robust
and efficient. Digital mode requires \eslmod{alphabet} augmentation.





 




% Table generated by autodoc -t esl_sq.c (so don't edit here, edit esl_sq.c:)
\begin{table}[hbp]
\begin{center}
{\small
\begin{tabular}{|ll|}\hline
\apisubhead{Text version of the \ccode{ESL\_SQ} object.}\\
\hyperlink{func:esl_sq_Create()}{\ccode{esl\_sq\_Create()}} & Description.\\
\hyperlink{func:esl_sq_CreateFrom()}{\ccode{esl\_sq\_CreateFrom()}} & Description.\\
\hyperlink{func:esl_sq_Grow()}{\ccode{esl\_sq\_Grow()}} & Description.\\
\hyperlink{func:esl_sq_GrowTo()}{\ccode{esl\_sq\_GrowTo()}} & Grows an \ccode{ESL\_SQ} to hold a seq of at least \ccode{n} residues.\\
\hyperlink{func:esl_sq_Copy()}{\ccode{esl\_sq\_Copy()}} & Make a copy of an \ccode{ESL\_SQ}.\\
\hyperlink{func:esl_sq_Reuse()}{\ccode{esl\_sq\_Reuse()}} & Description.\\
\hyperlink{func:esl_sq_Destroy()}{\ccode{esl\_sq\_Destroy()}} & Description.\\
\apisubhead{Digitized version of the \ccode{ESL\_SQ} object.}\\
\hyperlink{func:esl_sq_CreateDigital()}{\ccode{esl\_sq\_CreateDigital()}} & Description.\\
\hyperlink{func:esl_sq_CreateDigitalFrom()}{\ccode{esl\_sq\_CreateDigitalFrom()}} & Description.\\
\hyperlink{func:esl_sq_Digitize()}{\ccode{esl\_sq\_Digitize()}} & Description.\\
\hyperlink{func:esl_sq_Textize()}{\ccode{esl\_sq\_Textize()}} & Description.\\
\hyperlink{func:esl_sq_GuessAlphabet()}{\ccode{esl\_sq\_GuessAlphabet()}} & Guess alphabet type of a sequence.\\
\apisubhead{Other functions that operate on sequences.}\\
\hyperlink{func:esl_sq_SetName()}{\ccode{esl\_sq\_SetName()}} & Format and set a name of a sequence.\\
\hyperlink{func:esl_sq_SetAccession()}{\ccode{esl\_sq\_SetAccession()}} & Format and set the accession field in a sequence.\\
\hyperlink{func:esl_sq_SetDesc()}{\ccode{esl\_sq\_SetDesc()}} & Format and set the description field in a sequence.\\
\hyperlink{func:esl_sq_CAddResidue()}{\ccode{esl\_sq\_CAddResidue()}} & Add one residue (or terminal NUL) to a text seq.\\
\hyperlink{func:esl_sq_XAddResidue()}{\ccode{esl\_sq\_XAddResidue()}} & Add one residue (or terminal sentinel) to digital seq.\\
\hyperlink{func:esl_sq_GetFromMSA()}{\ccode{esl\_sq\_GetFromMSA()}} & Get a single sequence from an MSA.\\
\hyperlink{func:esl_sq_FetchFromMSA()}{\ccode{esl\_sq\_FetchFromMSA()}} & Fetch a single sequence from an MSA.\\
\hline
\end{tabular}
}
\end{center}
\caption{The \eslmod{sq} API.}
\label{tbl:sq_api}
\end{table}


\subsection{Getting data into and out of the ESL\_SQ object}

The \ccode{esl\_sq\_CreateFrom()} function is for creating a new
\ccode{ESL\_SQ} object from simple character strings for a sequence
and its name (also, optionally, an accession, description, and/or
secondary structure annotation string). This is to make it easier to
interface other code with Easel.

To get simple character strings back out of an \ccode{ESL\_SQ} object,
you can peek inside the object. The object is defined and documented
in \ccode{esl\_sqio.h}. It contains various information; the stuff you
need to know is:

\input{cexcerpts/sqio_sq}

The sequence itself is in \ccode{seq}. It contains \ccode{n}
residues. Residues are indexed \ccode{0..n-1} -- that is, \ccode{seq}
is a standard C string. 

A \ccode{name} string, containing the sequence name, is always
present. If the file format provided a sequence accession or a
sequence description, these strings are stored in \ccode{acc} and
\ccode{desc}, respectively; if either is unavailable for this
sequence, these strings are set to ``\verb+\0+''.

If optional per-residue secondary structure annotation is available
for the sequence, that annotation string is in \ccode{ss}, indexed the
same as \ccode{seq}; else, \ccode{ss} is \ccode{NULL}.

If the sequence has been digitized (see alphabet augmentation), the
\ccode{1..n} digital sequence is in \ccode{dsq}; else \ccode{dsq} is
\ccode{NULL}.

You can copy any of these strings, but don't alter them unless you
know what you're doing. Their memory is managed by the \ccode{ESL\_SQ}
object.
