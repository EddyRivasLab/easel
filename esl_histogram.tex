


\subsection{The histogram API}


\vspace{1em}
\begin{tabular}{ll}  \hline
   \multicolumn{2}{c}{\textbf{The \ccode{ESL\_HISTOGRAM} object}}\\
\ccode{esl\_histogram\_Create()}       & Create a new \ccode{ESL\_HISTOGRAM}. \\
\ccode{esl\_histogram\_CreateFull()}   & A \ccode{ESL\_HISTOGRAM} to keep all data samples   \\
\ccode{esl\_histogram\_Destroy()}      & Frees a \ccode{ESL\_HISTOGRAM}.\\
   \multicolumn{2}{c}{\textbf{Collecting and accessing data in a histogram}}\\
\ccode{esl\_histogram\_Add()}          & Add a sample to the histogram. \\
\ccode{esl\_histogram\_Sort()}         & Sorts samples in a full histogram. \\
\ccode{esl\_histogram\_GetScoreAtRank()}& Retrieve a high score.\\
\ccode{esl\_histogram\_Print()}        & Print a ``pretty'' ASCII histogram. \\
\ccode{esl\_histogram\_Plot()}         & Output a histogram in xmgrace XY format. \\
\ccode{esl\_histogram\_PlotSurvival()} & Output $P(X>x)$ in xmgrace XY format. \\
\ccode{esl\_histogram\_PlotTheory()}   & Output some expected distribution in xmgrace XY format. \\
\ccode{esl\_histogram\_PlotQQ()}       & Output a Q-Q plot in xmgrace XY format. \\
\ccode{esl\_histogram\_Goodness()}     & Evaluate goodness of fit between observed, expected.\\
   \multicolumn{2}{c}{\textbf{Declaring if and how data are censored}}\\
\ccode{esl\_histogram\_TrueCensoring()}      & Declare that data are censored. \\
\ccode{esl\_histogram\_VirtCensorByValue()}  & Declare that only data $>\phi$ are considered to be ``observed''. \\
\ccode{esl\_histogram\_VirtCensorByMass()}   & Declare that only the top $x$\% of the data are considered to be observed. \\
\ccode{esl\_histogram\_DeclareTailfitting()} & Declare that goodness of fit applies only to the observed tail.\\
\ccode{esl\_histogram\_DeclareRounding()}    & Declare that the data must be treated in bins.\\
\ccode{esl\_histogram\_SetExpect()}          & Set expected values \\
\hline
\end{tabular}
\vspace{1em}


\subsection{Collecting data}


\subsection{Declaring different kinds of observed and expected data situations}

If you do nothing else after collecting the data, the data are assumed
to be \emph{complete}: you observed all samples. If you fit to any
expected distribution, the expected distribution is assumed to
describe the complete data, and its parameters are to be fitted to the
complete data. This is the simplest, most obvious case.

Other situations may arise. The \eslmod{histogram} module was
designed to capture scores, and we are often most interested in the
highest, most significant scores. If a theoretically expected
distribution isn't a perfect fit to the observed data, we may want to
focus parameter fitting only on the rightmost (highest scoring) tail
of the observed data. This allows a fit to ignore low-scoring
outliers, for example. A similar case arises when we don't understand
the expected distribution of the complete data, but the tail seems to
follow a predictable form (such as an exponential tail); again we only
want to focus our fit only on the right tail.

In statistics, a \emph{censored} dataset arises when we only observe
samples with values in a particular range, and a known number $z$ of
samples have unknown values outside the observed range. (For instance,
imagine measuring the expected lifetime of light bulbs by checking
1000 bulbs every day for a year; we would only see some bulbs fail,
and for all the others all we know is that their failure time is $>1$
year.) Easel deals with all the cases it considers to be ``of
interest'' (from the perspective of score histograms, where the right
tail is the interesting part) by considering all such cases to be
examples of left-censored datasets. 

Specifically, Easel can deal with five different ``use cases'' in this
regard:

\begin{enumerate}
\item The data are complete, and fit to a distribution that describes the
      complete data. This is the usual case.

\item The data are complete, and they are fit to a distribution that
      describes the complete data, but the fitting and the
      goodness-of-fit test are performed only in the right
      (highest-scoring) tail with values $> \phi$. This is the case
      when we care more about the high-scoring right tail and we want
      to focus our parameter fitting there.

\item The data are complete, but they are fit to a distribution that
      only describes the right (highest scoring) tail $> \phi$, and
      the goodness-of-fit test is only performed on that tail. This
      is the case where we don't know a good distribution to fit all
      the data, but we can fit the tail to something (for instance, an
      exponential tail).

\item The data are censored such that no values $< \phi$ were
      recorded. The data are fit to a complete distribution,
      predicting the probability even of the censored (unobserved)
      values. Goodness of fit can only be evaluated in the observed
      data. This case is what is actually meant by censored data, but
      I have not seen it arise in sequence analysis.

\item The data are censored such that no values $< \phi$ were
      recorded. Only these observed data in the right tail are fit to
      a distribution (such as an exponential tail), and the goodness of fit
      is only evaluated in this observed tail. This case only
      arises by construction -- that is, it is possible to call existing
      Easel functions to set up this case, but I have not seen it
      arise in practical use.
\end{enumerate}

To set up the appropriate case, after you collect data samples in a
histogram, you \emph{declare} your problem by calling appropriate
functions in the histogram API.

To declare that a dataset in histogram \ccode{h} is truly
left-censored at some threshold $\phi$, with $z$ unobserved sample
values below that threshold, you call
\ccode{esl\_histogram\_TrueCensoring(h, z, phi)}. 

To declare that the dataset in histogram \ccode{h} should be
\emph{treated as if} it were left-censored, by ignoring all values
except those in the right tail, you have two options. You can either
set the threshold $\phi$ yourself with
\ccode{esl\_histogram\_VirtCensorByValue(h, phi)}, or you can specify
that you want to consider the rightmost fraction $t$ of the tail by
calling \ccode{esl\_histogram\_VirtCensorByMass(h, t)} (in which case
the appropriate threshold $t$ is identified). Note, however, that the
$\phi$ or $t$ threshold that you provide for virtual censoring is only
a suggestion.[...]

To declare that you are going to fit the ``observed'' data to a
distribution that only describes that observed tail, you call
\ccode{esl\_histogram\_DeclareTailfitting(h)}.

\subsubsection{Virtual censoring suggests but does not specify the threshold}

Because the data values are binned in the histogram object, it is
problematic to set any arbitrary threshold $\phi$ or $t$ for virtual
censoring, because that would mean splitting one data bin's total
counts into unobserved and observed counts. Unless we've kept the full
set of data values in addition to the binned counts, we can't do
that. Instead, the two virtual censoring functions treat $\phi$ or $t$
as \emph{suggestions}, and they set the actual $\phi$ threshold to be
the lower bound of the bin that the suggested $\phi$ or $t$ threshold
fell in. Thus the actual $\phi$ will always be $\geq$ the suggested
$\phi$, and the actual observed tail mass will be $\geq$ the suggested
$t$. If you need to know the actual values that got set in your
histogram \ccode{h}, you can access \ccode{h->phi} or calculate the
observed mass as \ccode{(float) (h->n - h->z) / (float) h->n}.

\subsubsection{More gory detail about the internals}

To implement the above, the \ccode{ESL\_HISTOGRAM} object tracks the
following variables:

\begin{description}
\item[\emcode{n}]  The number of data samples actually recorded in the histogram.
\item[\emcode{z}]  The number of data samples considered to be ``unobserved''.
\item[\emcode{Nc}] The number of data samples in the complete data.
\item[\emcode{No}] The number of data samples considered to be ``observed''.
\item[\emcode{Nx}] The number of data samples considered to be ``expected'', under 
                   the fitted distribution.
\end{description}

These get set for the five use cases as follows:

\begin{tabular}{rccccc}\hline
                       & \multicolumn{4}{c}{Case:}\\
                       & \textbf{1} & \textbf{2} &  \textbf{3}  & \textbf{4}    & \textbf{5} \\ \hline
Observed data:         &  Complete  & Complete   &  Complete    & Left-censored & Left-censored \\
Expected distribution: &  Complete  & Complete   &  Right tail  & Complete      & Right tail \\
Parameters fit to:     &  Complete  & Right tail &  Right tail  & Right tail    & Right tail \\
Goodness of fit on:    &  Complete  & Right tail &  Right tail  & Right tail    & Right tail \\\hline
\ccode{z}              &      0     &  \#$<\phi$ & \#$<\phi$    &  specified    &  specified \\
\ccode{Nc}             &      n     &     n      &     n        &    n+z        &    n+z     \\
\ccode{No}             &      n     &    n-z     &    n-z       &     n         &     n      \\
\ccode{Nx}             &      n     &     n      &    n-z       &    n+z        &     n      \\\hline
\end{tabular}








 

\subsection{Displaying data}







\subsection{Goodness of fit tests for observed data and expected distributions}

