
The sqio (sio) module contains routines for input from unaligned
sequence data files, such as FASTA files.

It implements two objects. A \ccode{ESL\_SQFILE} object works much
like an ANSI C \ccode{FILE}: it maintains information for an open
sequence file while it's being read. The \ccode{ESL\_SQ} object
contains information for a single sequence.

\subsection{Example of using the sqio API}

\begin{cchunk}
#include <easel/easel.h>
#include <easel/alphabet.h>
#include <easel/sqio.h>

int
main(void)
{
  ESL_SQ     *sq;
  ESL_SQFILE *sqfp;
  char        seqfile[] = "test.fa";

  if (esl_sqfile_OpenFASTA(filename, &sqfp) != ESL_OK) abort();
  sq = esl_sq_Create();

  while (esl_sio_ReadFASTA(sqfp, sq) == ESL_OK)
  {
    /* use the sequence for whatever you want */
    esl_sq_Reuse(sq);
  }
  esl_sqfile_Close(sqfp);
  esl_sq_Destroy(sq);
  return 0;
}
\end{cchunk}


A FASTA file named \ccode{filename} is opened for reading by calling
\ccode{esl\_sqfile\_OpenFASTA(filename, \&sqfp)}, which creates a new
\ccode{ESL\_SQFILE} and returns it through the \ccode{sqfp} pointer.
If the file doesn't exist, or can't be read properly, this function
returns a \ccode{ESL\_ENOFILE} error code; the caller should check for
this.

The file is then read one sequence at a time by calling
\ccode{esl\_sio\_ReadFASTA(sqfp, sq)}. This function returns
\ccode{ESL\_OK} if it read a new sequence, and leaves that sequence in
the \ccode{sq} object that the caller creates and provides.  When
there is no more data in the file, the \ccode{ReadFASTA()n} function
returns a \ccode{ESL\_EOF} code. If at any point the file does not
appear to be in the proper format, the \ccode{ReadFASTA()} function
returns a \ccode{ESL\_EFORMAT} code, and the application should deal
with this helpfully (though the example above does not). The API
currently doesn't make it terribly easy to figure out what went wrong,
but the application can look at \ccode{sqfp->linenumber} to obtain the
line number in the file at which the error occurred.

The \ccode{ESL\_SQ} object contains the sequence name, (optional)
description, and the sequence itself, as strings \ccode{sq->name},
\ccode{sq->desc}, and \ccode{sq->seq}, respectively. The length of the
sequence is in \ccode{sq->n}.

A \ccode{ESL\_SQ} object can be recycled and used for the next
sequence by calling \ccode{esl\_sq\_Reuse(sq)}. This is probably the
most efficient way to read a database one sequence at a
time. Alternatively, one could \ccode{esl\_sq\_Create()} as many
sequences as needed, in order to read in a set of sequences.

To clean up properly, a \ccode{ESL\_SQFILE} is closed with
\ccode{esl\_sqfile\_Close()}, and a \ccode{ESL\_SQ} that was created
is destroyed with \ccode{esl\_sq\_Destroy(sq)}.

