

Easel is a C code library for computational analysis of biological
sequences using probabilistic models.  Easel is used by HMMER and
Infernal, the software packages that the Pfam and Rfam protein and RNA
family databases rely on. Like any code library, Easel aims to make
applications more robust and easier to develop, by providing a set of
reusable, documented, and well-tested functions.

Easel's functionality includes generative probabilistic models of
sequences, phylogenetic models of evolution, bioinformatics tools for
sequence manipulation and annotation, numerical computing, and some
basic utilities. 

Easel's documentation includes tutorial examples to make it easy to
get started with any given Easel module. Easel is modular, to help you
extract individual files or functions for use in your own code,
without having to disentangle the entire library. Easel is freely
distributed as open source so that you can use and modify it for
anything, including commercial purposes.

\section{Quick start}

Let's assume you just want to take a quick tour. Maybe you're deciding
whether Easel is useful or not. You can compile Easel and try it out
without installing it.

After you obtain an Easel source tarball (from
\url{http://selab.janelia.org/easel}, for example), it should compile
out of the box on any UNIX, Linux, or Mac OS/X operating
system\footnote{You need a C compilation environment, which you might
not have. If not, install a C compiler first, such as the free GNU gcc
compiler, or see your local system guru.} with this incantation
(where \ccode{xxx} is a version number):

\begin{cchunk}
% tar zxf easel-xxx.tar.gz
% cd easel-xxx
% ./configure
% make
% make check
\end{cchunk}

The \ccode{make check} command is optional. It runs a battery of
quality control tests. All of these should pass. You should now see
\ccode{libeasel.a} in the directory.

If you decide you want to install Easel permanently, see the full
installation instructions in chapter~\ref{chapter:installation}.

Every source code module (that is, each \ccode{.c} file) ends with one
or more \esldef{driver programs}, including programs for unit tests
and benchmarks. These are \ccode{main()} functions that can be
conditionally included when the module is compiled. Among them is
always at least one \esldef{example driver} that shows you how to use
the module. You can find the example code in a module \eslmod{foo} by
searching the \ccode{esl\_foo.c} file for the tag
\ccode{eslFOO\_EXAMPLE}. To compile the example for module
\eslmod{foo} as a working program, do:

\begin{cchunk}
   % cc -o example -L. -I. -DeslFOO_EXAMPLE esl_foo.c -leasel -lm
\end{cchunk}

You might need to replace the standard \ccode{cc} with a different
compiler name, depending on your system. Linking to the standard math
library (\ccode{-lm}) may not be necessary, depending on what module
you're compiling, but it won't hurt. Replace \ccode{foo} with the name
of a module you want to play with, and you can compile any of Easel's
example drivers this way.

To run it, read the source code to see if it needs any command line
arguments, like the name of a file to open, then:

\begin{cchunk}
   % ./example <any args needed>
\end{cchunk}

You can edit the example driver to play around with it, if you like,
but it's better to make a copy of it in your own file (say,
\ccode{foo\_example.c}) so you're not changing Easel's code. When you
extract the code into a file, copy what's between the \ccode{\#ifdef
eslFOO\_EXAMPLE} and \ccode{\#endif /*eslFOO\_EXAMPLE*/} flags that
conditionally include the example driver (don't copy the flags
themselves).  Compile your example code like this:

\begin{cchunk}
   % cc -o foo_example -L. -I. foo_example.c esl_foo.c -leasel -lm
\end{cchunk}

That's it. Now you can compile Easel-based programs that you can
modify and play around with. If you're the type (like me) that prefers
to learn by example, you're on your way.  

The next question is, does Easel provide any functionality you're
interested in?  

\section{Overview of Easel's modules}

Each \ccode{.c} file in Easel corresponds to one Easel \esldef{module}.
A module consists of a group of functions for some
task. Table~\ref{tbl:module_list} shows a list of all the modules. For
example, the \eslmod{sqio} module can automatically parse many common
unaligned sequence formats, and the \eslmod{msa} module can parse many
common multiple alignment formats.

% documentation/esl-depends.pl
% can autoextract the module dependencies, including augmentations,
% but this table seems to work best as a simple list of modules.
%
\begin{table}
\begin{center}
\begin{tabular}{lll}\hline
\textbf{Module} & \textbf{Description} \\
  \multicolumn{2}{c}{\textbf{Foundation:}}\\
%
\eslmod{easel}           & Framework for using Easel & \\
%
  \multicolumn{2}{c}{\textbf{Core modules:}}\\
%
\eslmod{alphabet}        & Digitized biosequence alphabets        \\
\eslmod{dmatrix}         & Matrix algebra                         \\
\eslmod{fileparser}      & Input of simple token-based data files \\
\eslmod{getopts}         & Command line parsing                   \\
\eslmod{keyhash}         & Keyword hashing                        \\
\eslmod{msa}             & Multiple sequence alignment i/o        \\
\eslmod{random}          & Random number generator                \\
\eslmod{regexp}          & Regular expression matching            \\
\eslmod{sqio}            & Sequence file i/o                      \\
\eslmod{ssi}             & Indices for large sequence files       \\
\eslmod{stack}           & Pushdown stacks                        \\
\eslmod{stats}           & Foundation for statistics group        \\
\eslmod{stopwatch}       & Timing parts of programs               \\
\eslmod{vectorops}       & Vector operations                      \\
%
  \multicolumn{2}{c}{\textbf{Numerical computing group:}}\\
%
\eslmod{minimizer}       & Multidimensional optimization          \\
%
  \multicolumn{2}{c}{\textbf{Phylogeny group:}}\\
%
\eslmod{tree}         & Phylogenetic trees                   \\
\eslmod{distance}     & Phylogenetic distance calculations   \\
%
  \multicolumn{2}{c}{\textbf{Statistics group:}}             \\
%
\eslmod{histogram}    & Collecting/fitting data histograms   \\
\eslmod{dirichlet}    & Dirichlet densities.                 \\
\eslmod{exponential}  & Exponential densities.               \\
\eslmod{gamma}        & Gamma densities.                     \\
\eslmod{gev}          & Generalized extreme value (GEV) densities \\
\eslmod{gumbel}       & Gumbel densities.                    \\
\eslmod{hyperexp}     & Hyperexponential densities.          \\
\eslmod{mixdchlet}    & Mixture Dirichlet densities.         \\
\eslmod{mixgev}       & Mixture GEV densities.              \\
\eslmod{stretchexp}   & Stretched exponential densities.    \\
\eslmod{weibull}      & Weibull densities.                  \\
%
  \multicolumn{2}{c}{\textbf{Miscellaneous:}}\\
\eslmod{bioparse\_paml}  & PAML rate matrix datafiles        \\
\eslmod{ratematrix}      & Evolutionary rate matrices        \\
\eslmod{wuss}            & RNA structure annotation          \\
%
  \multicolumn{2}{c}{\textbf{Optional library interfaces:}}\\
%
interface\_gsl    & GNU Scientific Library          \\
interface\_lapack & LAPACK linear algebra library   \\
\hline
\end{tabular}
\end{center}
\caption{Overview of Easel's modules.}
\label{tbl:module_list}
\end{table}

Though you would normally use Easel as a C library
(\ccode{libeasel.a}) that you link with your code, Easel is also
designed to be sufficiently modular that you can grab individual
source files out of the library and use them directly in your own
code. For example, to get Easel's sequence file i/o API, for example,
you can take the sqio module (the C source \ccode{esl\_sqio.c} and the
header \ccode{esl\_sqio.h}), plus the obligatory Easel core
(\ccode{easel.c} and \ccode{easel.h}). Many of Easel's modules are
free-standing, and only depend on the foundation \eslmod{easel}
module. Some modules require a few other modules, but the total number
of modules you have to take to get any particular Easel API is always
small. Each module's documentation shows its required dependencies.

To minimize the number of modules you need to take to get some part of
Easel into your code, Easel uses a concept it calls
\emph{augmentation}. Each module provides a base functionality that is
as simple as possible, and which depends on as few other modules as
possible. When more powerful functionality would require additional
modules, where possible, Easel isolates that functionality and makes
it optional. You can \emph{augment} the module and activate these more
powerful optional abilities by providing the appropriate modules, or
you can leave the optional modules out.  For example, if you use only
the \eslmod{sqio} module, you get the ability to read unaligned
sequence files like FASTA or Genbank; but if you augment \eslmod{sqio}
with the \eslmod{msa} multiple alignment module, you gain the ability
to read individual sequences from multiple alignment files
sequentially. At compile-time, you declare (by means of
\ccode{\#define} flags in \ccode{easel.h}) what modules you've taken,
which defines what augmentations are possible. Each module's
documentation shows what optional augmentations are activated by other
modules. Of course, when Easel is used as a complete
\ccode{libeasel.a} library, all modules are fully augmented.




Many modules are organized around one or a few ``objects'' (usually a
C structure). Table~\ref{tbl:object_list} shows an overview of the
objects you might use (there are a few more objects that are used
internally). This quasi-object-orientation is used to simplify the
code design, focusing each module around a common datatype. 

\begin{table}
\begin{tabular}{lll}\hline
\textbf{Object}          & \textbf{Implemented in} & \textbf{Description}\\\hline
\ccode{ESL\_ALPHABET}    & \eslmod{alphabet}        & Digitized sequence alphabet\\
\ccode{ESL\_DMATRIX}     & \eslmod{dmatrix}         & 2D double-precision matrix for linear algebra \\
\ccode{ESL\_FILEPARSER}  & \eslmod{fileparser}      & Token-based input file parser\\
\ccode{ESL\_GETOPTS}     & \eslmod{getopts}         & Application configuration state\\
\ccode{ESL\_HISTOGRAM}   & \eslmod{histogram}       & Collecting/fitting data histograms\\
\ccode{ESL\_HYPEREXP}    & \eslmod{hyperexp}        & Hyperexponential distribution\\
\ccode{ESL\_KEYHASH}     & \eslmod{keyhash}         & Keyword hash table\\
\ccode{ESL\_MIXDCHLET}   & \eslmod{dirichlet}       & Mixture Dirichlet prior\\
\ccode{ESL\_MIXGEV}      & \eslmod{mixgev}          & Mixture generalized EVD\\
\ccode{ESL\_MSA}         & \eslmod{msa}             & Multiple sequence alignment\\
\ccode{ESL\_MSAFILE}     & \eslmod{msa}             & Multiple sequence alignment file parser\\
\ccode{ESL\_NEWSSI}      & \eslmod{ssi}             & New SSI index being created\\
\ccode{ESL\_OPTIONS}     & \eslmod{getopts}         & Configuration of a command-line option\\
\ccode{ESL\_PERMUTATION} & \eslmod{dmatrix}         & Permutation matrix used in linear algebra\\
\ccode{ESL\_RANDOMNESS}  & \eslmod{random}          & Random number generator\\
\ccode{ESL\_REGEXP}      & \eslmod{regexp}          & Regular expression pattern-matching machine\\
\ccode{ESL\_SQ}          & \eslmod{sqio}            & DNA/RNA/protein sequence data\\
\ccode{ESL\_SQFILE}      & \eslmod{sqio}            & Biosequence file parser (unaligned)\\
\ccode{ESL\_SSI}         & \eslmod{ssi}             & An SSI index being used\\
\ccode{ESL\_STACK}       & \eslmod{stack}           & Pushdown stack\\
\ccode{ESL\_STOPWATCH}   & \eslmod{stopwatch}       & Timer for parts of a program\\
\hline
\end{tabular}
\caption{Overview of Easel's objects.}
\label{tbl:object_list}
\end{table}

A pitfall of putting data into complex, custom datatypes is that this
can impede code reuse. You don't want to understand what's in someone
else's object, and you probably have your own favorite ways of
organizing data.  Easel assumes you are only going to use Easel
objects to call Easel functions, so you will build simple interfaces
to exchange data between your code and Easel. Thus, there are always
obvious ways to create new Easel objects from elemental data types,
and to extract elemental data types from Easel objects. Easel objects
are translucent, if not transparent; often, some of their internal
data fields are stable and documented, and you are encouraged to reach
into an object and access elemental data directly.


