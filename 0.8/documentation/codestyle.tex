
This chapter describes Easel, from an Easel developer's perspective:
how a module's source code is organized and documented. It provides a
guide for implementing new modules and for reading existing ones.

Like Emerson, Easel is not overly concerned with
consistency. Nonetheless, Easel modules are organized in similar ways,
and similar functions are channeled towards common \esldef{interfaces}
that provide guidelines for a shared behavior and a shared naming
convention. These shared patterns have been established as a result of
experience. Common interface designs also help reduce the apparent
complexity of the library, by making many functions behave in
predictable ways, so it is easier to learn for a programmer to use new
Easel modules.

Some of the language in this chapter is intended to convey specific
legalistic meaning:

\begin{itemize} 
\item \emph{Must} indicates a requirement: noncompliance means the
code is broken.

\item \emph{Shall} means a requirement for \emph{new and revised}
Easel code, but older Easel code is ``grandfathered in'' and may not
necessarily meet this spec. (The analogy is in how housing
construction codes are used to drive evolutionary improvement. New
housing codes don't force all old construction to be immediately
renovated to meet the new requirements, but new homes and renovations
do need to meet modern requirements. Over time, this means housing
evolves and improves.)\footnote{If you're familiar with RFC2119, it
considers ``must'', ``required'', and ``shall'' to mean the same
thing, unlike Easel. RFC2119 was designed for static
specifications. It does not define terms suitable for an evolving
specification.}

\item \emph{Should} means a best practice, the preferred or
recommended choice among compliant alternatives.

\item \emph{May} means a non-preferred or less frequent choice among
compliant alternatives.

\item Any other statement, such as ``Each module consists of three
files'' indicates a best practice that is used with few or no
exceptions.
\end{itemize}




%%%%%%%%%%%%%%%%%%%%%%%%%%%%%%%%%%%%%%%%%%%%%%%%%%%%%%%%%%%%%%%%
\section{Nomenclature}
%%%%%%%%%%%%%%%%%%%%%%%%%%%%%%%%%%%%%%%%%%%%%%%%%%%%%%%%%%%%%%%%

   \subsection{Module names}

Module names should be ten characters or less.\footnote{sqc assumes
  this in its output formatting, for example.}

If a module is organized around an object, they should have the same
name: for example, the \eslmod{alphabet} module implements the
\ccode{ESL\_ALPHABET} object.

The module should also have a three-letter abbreviation. For instance,
the \eslmod{alphabet} module is also known by the abbreviation
\eslmod{abc}. The three-letter code is used to construct shorter
function and macro names.

Each module consists of three files: a .c source code file, a .h
header file, and a .tex documentation file. These filenames are
constructed from the module name. For example, the \eslmod{dmatrix}
module is implemented in \ccode{esl\_dmatrix.c},
\ccode{esl\_dmatrix.h}, and \ccode{esl\_dmatrix.tex}.


   \subsection{Functions}

Function names are tripartite, constructed as
\ccode{esl\_\{tag\}\_\{name\}}.  The \ccode{\{tag\}} is either an
object name, the module's full name, or the module's three-letter
abbreviation. The \ccode{\{name\}} can be anything, but some names are
standard and indicate the use of a common interface.

In a module that implements one or more objects, different tags are
used to indicate functions that operate directly on objects via common
interfaces, versus other functions in the exposed API.  Functions that
act directly on an object using a common interface should be named by
the object name, such as \ccode{esl\_alphabet\_Create()}.\ccode{This
is a clumsy C version of what C++ would do with namespaces, object
methods, and constructors/destructors.} Other functions in the API of
a module should be named by the three-letter code, such as
\ccode{esl\_abc\_Digitize()}.

In modules that implement no objects, a single tag is used. The
\ccode{\{tag\}} can either be the full module name or the three-letter
abbreviation. Whichever is used, that choice should be used
consistently in all exposed functions in a given module.

Only exposed (\ccode{extern}) functions must follow these rules. In
general, private (\ccode{static}) functions can have any name.

Sometimes essentially the same function must be provided for different
data types, in which case one-letter prefixes are used to indicate
datatype:

\begin{tabular}{ll}
\ccode{C} & \ccode{char} type, or a standard C string \\
\ccode{X} & \ccode{ESL\_DSQ} type, or an Easel digitized sequence\\
\ccode{I} & \ccode{int} type \\
\ccode{F} & \ccode{float} type \\
\ccode{D} & \ccode{double} type \\
\end{tabular}

For example, \eslmod{vectorops} uses this convention heavily;
\ccode{esl\_vec\_FNorm()} normalizes a vector of floats and
\ccode{esl\_vec\_DNorm()} normalizes a vector of doubles.  A second
example is in \eslmod{random}, which provides routines for shuffling
either text strings or digitized sequences, such as
\ccode{esl\_rnd\_CShuffle()} and \ccode{esl\_rnd\_XShuffle()}.

   \subsection{Macros}

Macro names start with \ccode{ESL\_} and are all upper case, as in
\ccode{ESL\_EXCEPTION()}.

Macros that only become available when debugging hooks are activated
at compile time are prefixed\ccode{ESL\_D} and end with the debugging
level, as in the \ccode{ESL\_DPRINTF2()} macro that is activated at
level 2 debugging.

   \subsection{Constants}

Constants (including both \ccode{\#define}'s and \ccode{enum}'s) start
with \ccode{esl} followed by all upper case, as in
\ccode{eslINFINITY}.

Error codes are \ccode{eslE} followed by upper case, as in
\ccode{eslEDIVZERO}, except for \ccode{eslOK} and \ccode{eslFAIL}.

Magic mathematical constants start with \ccode{eslCONST\_} followed by
an upper case name, as in \ccode{eslCONST\_PI}.





%%%%%%%%%%%%%%%%%%%%%%%%%%%%%%%%%%%%%%%%%%%%%%%%%%%%%%%%%%%%%%%%
\section{The .c source file}
%%%%%%%%%%%%%%%%%%%%%%%%%%%%%%%%%%%%%%%%%%%%%%%%%%%%%%%%%%%%%%%%

The .c file starts with a block comment that contains a line stating
the module's purpose; a table of contents listing the sections the
file is organized into (see below); an incept stamp (initials and
date) to help track the provenance of this file \footnote{The physical
place where the code was started and the music we were listening to
may also be nostalgically noted on the incept line.}; and an
\ccode{SVN \$Id\$} tag that is filled in by Subversion.  The name of
the file should not appear in this comment. It is redundant, and
becomes just one more thing to change if we ever rename the module.

Included headers are next. The \ccode{esl\_config.h} header must
always be first, because it contains platform-independent
configuration code that may affect even the standard library header
files. Standard headers like \ccode{stdio.h} come next, then Easel's
main header \ccode{easel.h}; then headers of any other Easel modules
this module depends on; then any headers for modules this module can
be augmented with, surrounded by appropriate \ccode{\#ifdef}'s; then
the module's own header. For example, the \ccode{\#include}'s in the
\eslmod{msa} module look like:

\begin{cchunk}
#include <esl_config.h>

#include <stdio.h>
#include <stdlib.h>
#include <string.h>
#include <ctype.h>

#include <easel.h>
#ifdef eslAUGMENT_KEYHASH
#include <esl_keyhash.h>
#endif
#ifdef eslAUGMENT_ALPHABET
#include <esl_alphabet.h>
#endif
#ifdef eslAUGMENT_SSI
#include <esl_ssi.h>
#endif
#include <esl_msa.h>
\end{cchunk}

The rest of the file is split into sections, which are numbered and
given one-line titles that appear in the table of contents, comments
in front of each code section in the .c file, and comments in front of
that section's declarations in the .h file. Because of the numbering
and titling, it is easy to locate a particular section of code by
searching on the number (and possibly the title).  A common section
structure includes the following, in this order:

\begin{description}

\item[\textbf{The \ccode{FOOBAR} object.}]

  These routines create and destroy objects, setting and getting
  information in objects, cloning and copying objects (i.e.  copies
  with or without new memory allocation), or comparing objects for
  equality. Common interfaces include:

  \begin{itemize}
      \item \ccode{foobar\_Create*()}
      \item \ccode{foobar\_Set*()}
      \item \ccode{foobar\_Get*()}
      \item \ccode{foobar\_Add*()}
      \item \ccode{foobar\_Copy*()}
      \item \ccode{foobar\_Clone*()}
      \item \ccode{foobar\_Compare*()}
      \item \ccode{foobar\_Destroy*()}
  \end{itemize}

  Input streams are opened/closed rather than created/destroyed.
  Common interfaces for input stream objects include:

  \begin{itemize}
      \item \ccode{foobar\_Open*()}
      \item \ccode{foobar\_Close*()}
  \end{itemize}

  Some objects are reusable, to save cycles of malloc/free.
  Common interfaces for reusable objects include:

  \begin{itemize}
      \item \ccode{foobar\_Reuse*()}
      \item \ccode{foobar\_Grow*()}
      \item \ccode{foobar\_GrowTo()}
      \item \ccode{foobar\_Shrink()}
  \end{itemize}

  All routines in the object section follow one of these common
  interfaces. Anything else is in the Exposed API section.

\item[\textbf{Debugging/dev code for the object.}]

  Most objects can be validated or dumped to an output stream
  for inspection.

  \begin{itemize}
      \item \ccode{foobar\_Validate*()}
      \item \ccode{foobar\_Dump*()}
  \end{itemize}

\item[\textbf{The rest of the base API.}]

  Everything else that is part of the API for this module in its
  baseline (unaugmented) form. Usually these functions don't follow
  any particular common interface.  These might be split across
  multiple sections.

  API functions typically have abbreviated names like
  \ccode{esl\_foo\_DoSomething()} instead of the object name
  \ccode{esl\_foobar\_SetSomething()}.

\item[\textbf{Augmented API, if any.}]

  Any functions that are only available with one or more augmentations
  are split into separate sections. 

\item[\textbf{Private functions.}]

  Easel isn't rigorous about where private (non-exposed) functions go,
  but they often go in a separate section in about the middle of the
  \ccode{.c} file, after the API and before the drivers.

\item[\textbf{Stats driver (if any).}]

  Some modules have a \ccode{main()} for collecting statistics and
  generating plots analyzing the scientific performance of a module.

\item[\textbf{Benchmark driver (if any).}]

  Some modules have a \ccode{main()} for running timing benchmarks.

\item[\textbf{Regression driver (if any).}]

  Some modules have a \ccode{main()} for comparing the performance of
  a module to previous instantiations of it in Easel, in other
  codebases of ours (SQUID, for example), or in other available
  libraries (GSL, for example).

\item [\textbf{Unit tests.}]
 
  The unit tests are internal controls that test that the module's API
  works as advertised.

\item [\textbf{Test driver.}]

  All modules have an automated test driver is a \ccode{main()} that
  runs the unit tests.
 
\item [\textbf{Example code.}]

  All modules have at least one \ccode{main()} showing an example of
  how to use the main features of the module.

\item [\textbf{Copyright/license information.}]

  Easel is freely available under the Janelia Software License (JSL),
  but these stamps aren't added until a distribution is
  packaged. Instead, each file ends with a \ccode{@LICENSE@} tag. This
  placeholder is automatically replaced by the correct license
  statement at packaging time. This gives us the ability to package
  specially licensed versions, if needed.

\end{description}


%%%%%%%%%%%%%%%%%%%%%%%%%%%%%%%%%%%%%%%%%%%%%%%%%%%%%%%%%%%%%%%%
\section{The .h header file}
%%%%%%%%%%%%%%%%%%%%%%%%%%%%%%%%%%%%%%%%%%%%%%%%%%%%%%%%%%%%%%%%

...

%%%%%%%%%%%%%%%%%%%%%%%%%%%%%%%%%%%%%%%%%%%%%%%%%%%%%%%%%%%%%%%%
\section{The .tex documentation file}
%%%%%%%%%%%%%%%%%%%%%%%%%%%%%%%%%%%%%%%%%%%%%%%%%%%%%%%%%%%%%%%%

...


%%%%%%%%%%%%%%%%%%%%%%%%%%%%%%%%%%%%%%%%%%%%%%%%%%%%%%%%%%%%%%%%
\section{Function names and common interfaces}
%%%%%%%%%%%%%%%%%%%%%%%%%%%%%%%%%%%%%%%%%%%%%%%%%%%%%%%%%%%%%%%%

Most Easel functions have three part names like
\ccode{esl\_foo\_BarBaz()}, where the \ccode{esl} prefix defines an
Easel-specific namespace (to keep Easel functions from clashing with
the names of an application's functions), \ccode{foo} is the
lower-case base name of the Easel object that is acted on (or the
module that this function is in), and \ccode{BarBaz()} describes what
the function does.  If there are two or more functions
\ccode{esl\_foo\_BarBaz()}, \ccode{esl\_bar\_BarBaz()}, etc., then we
can talk about the \ccode{\_BarBaz()} interface. 

Conversely, if there is a \ccode{\_BarBaz()} interface, then any
function named something like \ccode{esl\_foo\_BarBaz()} or
\ccode{esl\_foo\_BarBazQux()} follows it. For example,
\ccode{esl\_sq\_Create()} is going to create a \ccode{ESL\_SQ}
sequence object, following interface guidelines for \ccode{\_Create()}
functions.

\subsection{Creating and destroying new objects}

Most Easel objects are allocated and free'd by
\ccode{\_Create()/\_Destroy()} interface. Creating an object often
just means allocating space for it, so that some other routine can
fill data into it. It does not necessarily mean that the object
contains valid data.

In summary:

\begin{sreapi}
\hypertarget{ifc:Create} 
{\item[\_Create(N)]}

A \ccode{\_Create()} interface takes any necessary initialization or
size information as arguments (there may not be any), and it returns a
pointer to the newly allocated object. If an (optional) number of
elements \ccode{N} is provided, this specifies the number of elements
that the object is going to contain (thus, creating a fixed-size
object; contrast the \ccode{\_CreateGrowable()} interface).  In the
event of any failure, the procedure throws \ccode{NULL}.

\hypertarget{ifc:Destroy} 
{\item[\_Destroy(obj)]}
A \ccode{\_Destroy()} interface takes an object pointer as an
argument, and frees all the memory associated with it.
\end{sreapi}

For example:
\begin{cchunk}
   ESL_SQ *sq;
   sq = esl_sq_Create();
   esl_sq_Destroy(sq);
\end{cchunk}



\subsubsection{Open input streams}

Some objects (such as \ccode{ESL\_SQFILE} and \ccode{ESL\_MSAFILE})
correspond to open input streams -- usually an open file, but possibly
reading from a pipe. Such objects are \ccode{\_Open()}'ed and
\ccode{\_Close()'d}, not created and destroyed.

Input stream objects have to be capable of handling normal failures,
because of bad user input. Input stream objects contain an
\ccode{errbuf[eslERRBUFSIZE]} field to capture informative parse error
messages. 

In summary:

\begin{sreapi}
\hypertarget{ifc:Open} 
{\item[\_Open(file, formatcode, \&ret\_obj)]}

Opens the \ccode{file}, which is in a format indicated by
\ccode{formatcode} for reading; return the open input object in
\ccode{ret\_obj}. A \ccode{formatcode} of 0 typically means unknown,
in which case the \ccode{\_Open()} procedure attempts to autodetect
the format. If the \ccode{file} is \ccode{"-"}, the object is
configured to read from the \ccode{stdin} stream instead of opening a
file. If the \ccode{file} ends in a \ccode{.gz} suffix, the object is
configured to read from a pipe from \ccode{gzip -dc}. Returns
\ccode{eslENOTFOUND} if \ccode{file} cannot be opened, and
\ccode{eslEFORMAT} if autodetection is attempted but the format cannot
be determined.

\hypertarget{ifc:Close} 
{\item[\_Close(obj)]}

Closes the input stream \ccode{obj}. Returns \ccode{void}.
\end{sreapi}


For example:

\begin{cchunk}
    char        *seqfile = "foo.fa";
    ESL_SQFILE  *sqfp;

    esl_sqio_Open(seqfile, eslSQFILE_FASTA, NULL, &sqfp);
    esl_sqio_Close(sqfp);
\end{cchunk}



\subsubsection{Growable objects}

Some objects need to be reallocated and expanded during their use.
These objects are called \esldef{growable}.

In some cases, the whole purpose of the object is to have elements
added to it, such as \ccode{ESL\_STACK} (pushdown stacks) and
\ccode{ESL\_HISTOGRAM} (histograms). In these cases, the normal
\ccode{\_Create()} interface performs an initial allocation, and the
object keeps track of both its current contents size (often
\ccode{obj->N}) and the current allocation size (often
\ccode{obj->nalloc}). 

In some other cases, objects might be either growable or not,
depending on how they're being used. This happens, for instance, when
we have routines for parsing input data to create a new object, and we
need to dynamically reallocate as we go because the input doesn't tell
us the total size when we start. For instance, with \ccode{ESL\_TREE}
(phylogenetic trees), sometimes we know exactly the size of the tree
we need to create (because we're making a tree ourselves), and
sometimes we need to create a growable object (because we're reading a
tree from a file). In these cases, the normal \ccode{\_Create()}
interface creates a static, nongrowable object of known size, and a
\ccode{\_CreateGrowable()} interface specifies an initial allocation
for a growable object.

Easel usually handles its own reallocation of growable objects. For
instance, many growable objects have an interface called something
like \ccode{\_Add()} or \ccode{\_Push()} for storing the next element
in the object, and this interface will deal with increasing allocation
size as needed.  In a few cases, a public \ccode{\_Grow()} interface
is provided for reallocating an object to a larger size, in cases
where a caller might need to grow the object itself. \ccode{\_Grow()}
only increases an allocation when it is necessary, so that a caller
can call \ccode{\_Grow()} before every attempt to add a new
element. An example of where a public \ccode{\_Grow()} interface is
generally provided is when an object might be input from different
file formats, and an application may need to create its own
parser. Although creating an input parser requires familiarity with
the Easel object's internal data structures, at least the
\ccode{\_Grow()} interface frees the caller from having to understand
its memory management.

Growable objects waste memory, because they are overallocated in order
to reduce the number of calls to \ccode{malloc()}.  The wastage is
bounded (to a maximum of two-fold, for the default doubling
strategies, once an object has exceeded its initial allocation size)
but nonetheless may not always be tolerable.  A \ccode{\_Shrink()}
interface optimizes the memory usage in an object and converts it to a
nongrowable, fixed-size form. 

In summary: 

\begin{sreapi}
\hypertarget{ifc:CreateGrowable}
{\item[\_CreateGrowable(nalloc)]}

A \ccode{\_CreateGrowable()} interface creates a growable
\ccode{obj}. If a size argument like \ccode{nalloc} is provided, it
specifies an initial allocation size, not the number of elements.

\hypertarget{ifc:Grow}
{\item[\_Grow(obj)]}

Check to see if \ccode{obj} can hold another element. If not, increase
the allocation, according to internally stored rules on reallocation
strategy (usually, by doubling).

\hypertarget{ifc:Shrink}
{\item[\_Shrink(obj)]}

Optimize the memory usage in growable object \ccode{obj}, converting
it to a fixed-size, nongrowable object.

\hypertarget{ifc:CreateCustom}
{\item[\_CreateCustom(my\_nalloc)]}

A \ccode{\_CreateCustom()} interface to a growable object might enable
a caller to alter the object's default initial allocation size, for
cases where the \ccode{Create()} interface uses a hardcoded default
size and this default size might not suffice for all applications.

\hypertarget{ifc:SetGrowth}
{\item[\_SetGrowth(obj, nfactor)]}

Growable objects are usually reallocated by increasing the current
allocation by some factor. The default reallocation factor is usually
2, so that objects usually grow by doubling. The API may provide a
\ccode{\_SetGrowth()} interface for changing the reallocation factor
for \ccode{obj} to \ccode{nfactor} at any time.
\end{sreapi}




\subsubsection{Reusable objects}

Memory allocation is computationally expensive. An application needs
to minimize \ccode{malloc()/free()} calls in performance-critical
regions. In loops where one \ccode{\_Destroy()}'s an old object only
to \ccode{\_Create()} the next one, such as a sequential input loop
that processes objects from a file one at a time, one generally wants
to \ccode{\_Reuse()} the same object instead:

\begin{sreapi}
\hypertarget{ifc:Reuse}
{\item[\_Reuse(obj)]}

A \ccode{\_Reuse()} interface takes an existing object and
reinitializes it as a new object, while reusing as much memory as
possible. It replaces a \ccode{\_Destroy()/\_Create()} pair. A
\ccode{\_Reuse()} interface does not care whether the object was
originally created by a \ccode{\_Create()} or a \ccode{\_Inflate()}
call (see below), or whether the object is growable or not (if it was
growable, it remains growable).
\end{sreapi}

For example:

\begin{cchunk}
   ESL_SQFILE *sqfp;
   ESL_SQ     *sq;

   esl_sqfile_Open(\"foo.fa\", eslSQFILE_FASTA, NULL, &sqfp);
   sq = esl_sq_Create();
   while (esl_sqio_Read(sqfp, sq) == eslOK)
    {
       /* do stuff with this sq */
       esl_sq_Reuse(sq);
    }
   esl_sq_Destroy(sq);
\end{cchunk}


\subsubsection{Stack-allocated objects}

Most of Easel's objects are allocated on the heap; that is,
\ccode{malloc()'ed} memory accessed exclusively via pointers. Less
often, an interface may allow the shell of an object to be allocated
on the stack instead, using \ccode{\_Inflate(),\_Deflate()}.  The only
difference between \ccode{\_Inflate(),\_Deflate()} and
\ccode{\_Create(),\_Destroy()} is whether the object shell itself
needs to be allocated, or not:

\begin{sreapi}
\hypertarget{ifc:Inflate}
{\item[\_Inflate()]}

An \ccode{\_Inflate(\&foo)} interface is the on-stack version of
\ccode{foo = \_Create()}. The contents of an existing object shell
are allocated just as in \ccode{\_Create()}.

\hypertarget{ifc:Deflate}
{\item[\_Deflate()]}

A \ccode{\_Deflate(\&foo)} interface is the on-stack version of
\ccode{\_Destroy(foo)}. The contents of an object shell are free'd,
just as in \ccode{\_Destroy()}.
\end{sreapi}

For example:

\begin{cchunk}
   ESL_SQ  sq;

   esl_sq_Inflate(&sq);
   esl_sq_Deflate(&sq);
\end{cchunk}


\subsection{Manipulating and accessing objects}

\begin{sreapi}
\hypertarget{ifc:Copy}
{\item[\_Copy(src, dest)]}

Copies \ccode{src} object into \ccode{dest}, where the caller has
already created an appropriately allocated and empty \ccode{dest}
object. Returns \ccode{ESL\_OK} on success; throws
\ccode{ESL\_EINCOMPAT} if the objects are not compatible (for example,
two matrices that are not the same size).

Note that the order of the arguments is always \ccode{src}
$\rightarrow$ \ccode{dest} (unlike the C library's \ccode{strcpy()}
convention, which is the opposite order).

\hypertarget{ifc:Duplicate}
{\item[\ccode{\_Duplicate(obj)}] }

Creates and returns a pointer to a duplicate of \ccode{obj}.
Equivalent to (and is a shortcut for) \ccode{dest = \_Create();
\_Copy(src, dest)}. Caller is responsible for free'ing the duplicate
object, just as if it had been \ccode{\_Create}'d. Throws NULL if
allocation fails.

\hypertarget{ifc:Set}
\item[\ccode{\_Set*(obj, value...)}]

Initializes some value(s) in \ccode{obj} to
\ccode{value}. \ccode{\_Set} functions have some appropriate longer
name, like \ccode{\_SetZero()} (set something in an object to
zero(s)), or \ccode{esl\_dmatrix\_SetIdentity()} (set a dmatrix to an
identity matrix).

\hypertarget{ifc:Get}
{\item[\ccode{\_Get*(obj, ..., \&ret\_value)}]}

Retrieves some specified data from \ccode{obj}, leave it in
\ccode{ret\_value}.

\hypertarget{ifc:Compare}
{\item[\ccode{\_Compare*(obj1, obj2...)}]}

Compares \ccode{obj1} to \ccode{obj2}. Returns \ccode{eslOK} if the
contents are judged to be identical, and \ccode{eslFAIL} if they
differ. When the comparison involves floating point scalar
comparisons, a fractional tolerance argument \ccode{tol} is also
passed. 

% examples: esl_dmatrix_Compare(), static msa_compare(),
%   esl_tree_Compare(), esl_vec_{DFI}Compare().
\end{sreapi}



\subsection{Debugging, testing, development interfaces}

\begin{sreapi}
\hypertarget{ifc:Dump}
{\item[\ccode{\_Dump*(FILE *fp, obj...)}]}

Prints the internals of an object in human-readable, easily parsable
tabular ASCII form. Useful during debugging and development to view
the entire object at a glance. Returns \ccode{eslOK} on success.

\hypertarget{ifc:Validate}
{\item[\ccode{\_Validate*(obj, errbuf...)}]}

Checks that the internals of \ccode{obj} are all right. Returns
\ccode{eslOK} if they are, and returns \ccode{eslFAIL} if they
aren't. Additionally, if the caller provides a non-\ccode{NULL}
message buffer \ccode{errbuf}, on failure, an informative message
describing the reason for the failure is formatted and left in
\ccode{errbuf}. If the caller provides this message buffer, it must
allocate it for at least \ccode{p7\_ERRBUFSIZE} characters.

Because a \ccode{\_Validate()} call is primarily intended for
debugging, failures are classified as normal (returned) errors.
(Throwing exceptions instead would give the caller no flexibility in
dealing with failed validations.) The caller can then print an
appropriately informative failure message, using the \ccode{errbuf},
for example. (A \ccode{\_Validate()} call might also be included in
production code, so it needs to use the \ccode{errbuf} mechanism to
return its informative failure message; it cannot, for instance, call
\ccode{esl\_fatal()} or the like directly, because that would mean
crashing out of production code.)

\end{sreapi}



\subsection{Other interfaces}

\begin{sreapi}
\hypertarget{ifc:Describe}
{\item[\ccode{\_DescribeXXX()}]}

Given an internal code (an \ccode{enum} or \ccode{\#define} constant),
return a pointer to an informative string, for diagnostics and other
output. The string is static.
\end{sreapi}





%%%%%%%%%%%%%%%%%%%%%%%%%%%%%%%%%%%%%%%%%%%%%%%%%%%%%%%%%%%%%%%%
\section{An Easel function}
%%%%%%%%%%%%%%%%%%%%%%%%%%%%%%%%%%%%%%%%%%%%%%%%%%%%%%%%%%%%%%%%

...


%%%%%%%%%%%%%%%%%%%%%%%%%%%%%%%%%%%%%%%%%%%%%%%%%%%%%%%%%%%%%%%%
\section{An Easel object}
%%%%%%%%%%%%%%%%%%%%%%%%%%%%%%%%%%%%%%%%%%%%%%%%%%%%%%%%%%%%%%%%

...


%%%%%%%%%%%%%%%%%%%%%%%%%%%%%%%%%%%%%%%%%%%%%%%%%%%%%%%%%%%%%%%%
\section{Unit tests}
%%%%%%%%%%%%%%%%%%%%%%%%%%%%%%%%%%%%%%%%%%%%%%%%%%%%%%%%%%%%%%%%

Unit tests are called by the test driver (see below). There is often
one unit test assigned to each exposed function in the API. Sometimes,
it makes sense to test several exposed functions in a single unit test
function. 

Like the test driver, the unit test section is wrapped in a
\ccode{\#ifdef eslFOO\_TESTDRIVE}, so it is conditionally compiled only
for testing purposes.

A unit test for \ccode{esl\_foo\_Baz()} is named \ccode{static void
utest\_Baz()}. 

Upon success, unit tests return void. The test driver just calls 
unit tests one after another.

Upon any failure, a unit test calls \ccode{esl\_fatal()} with an error
message, and terminates. It should not use any other error-catching
mechanism; it aids debugging if the test program terminates
immediately, using a single function that we can easily breakpoint at
(\ccode{break esl\_fatal} in GDB). It must not use \ccode{abort()}, in
particular, because this will screw up the output of scripts running
automated tests in \ccode{make check} and \ccode{make dcheck}. These
scripts trap the \ccode{stderr} from \ccode{esl\_fatal()} correctly.
It must not use \ccode{exit(1)} either, because that leaves no error
message, so someone running a test program on the command line can't
easily tell that it failed.

Every function, procedure, and macro in the exposed API shall be
tested by one or more unit tests. The unit tests aim for complete code
coverage. This is measured by code coverage tests using GCOV.

Unit tests shall attempt to deliberately generate exceptions and
failures, and test that the appropriate error code is returned.  This
test code must be enclosed in \ccode{\#ifdef eslTEST\_THROWING} tags.
Exception testing cannot be done with the default fatal exception
handler installed, because exceptions would cause the program to
terminate with a nonzero code, and this would look like a test failure
to an automated test harness.

Unit tests shall test all possible combinations of augmentations that
may affect a function.

\subsection{Using GCOV} 

We use the GNU \textsc{gcov} program to measure code
coverage. \textsc{gcov} works best with unoptimized code, so that the
optimizer doesn't combine any lines of code, and it is only compatible
with the \textsc{gcc} compiler. An example of measuring code coverage
for the \eslmod{msa} module in full library configuration:

\begin{cchunk}
  % make distclean
  % ./configure --enable-debugging
  % make
  % gcc -fprofile-arcs -ftest-coverage -g -Wall -o test -L. -I. -DeslMSA_TESTDRIVE esl_msa.c -leasel -lm
  % ./test
  % gcov esl_msa.c
  File `esl_msa.c'
  Lines executed:65.30% of 1317
  esl_msa.c:creating `esl_msa.c.gcov'
\end{cchunk}

The file \ccode{esl\_msa.c.gcov} contains an annotated source listing
of the \ccode{.c} file, showing which lines were and weren't covered
by the test suite.

The \ccode{coverage\_test.pl} script in testsuite automates coverage
testing for all Easel modules. To run it:

\begin{cchunk} 
   % testsuite/coverage_test.pl
\end{cchunk}



%%%%%%%%%%%%%%%%%%%%%%%%%%%%%%%%%%%%%%%%%%%%%%%%%%%%%%%%%%%%%%%%
\section{Drivers}
%%%%%%%%%%%%%%%%%%%%%%%%%%%%%%%%%%%%%%%%%%%%%%%%%%%%%%%%%%%%%%%%

Each module contains one or more \ccode{main()} functions for testing
and example purposes, enclosed in \ccode{\#ifdef}'s to control
conditional compilation.

There are currently five different types of \ccode{main()} functions
in Easel modules:

\begin{description} 

\item[\textbf{Automated test driver.}] Each module has one (and only
  one) \ccode{main()} that runs the unit tests and any other automated
  for the module. The test driver is compiled and run by the testsuite
  in \ccode{testsuite/testsuite.sqc} when one does a \ccode{make
  check} on the package. It is also run by several of the automated
  tools used in development, including the coverage (\ccode{gcov}) and
  memory (\ccode{valgrind}) tests.  The test driver is enclosed by
  \ccode{\#ifdef eslFOO\_TESTDRIVE} tags.

\item[\textbf{Regression/comparison tests.}] (Optional.) These tests
  link to at least one other existing library that provides comparable
  functionality, such as the old SQUID library or the GNU Scientific
  Library, and test that Easel's functionality performs at least as
  well as the 'competition'. These tests are run on demand, and not
  included in automated testing, because the other libraries may only
  be present on a subset of our development machines. They are
  enclosed by \ccode{\#ifdef eslFOO\_REGRESSION} tags.

\item[\textbf{Benchmark tests.}] (Optional.) These tests run a
  standardized performance benchmark and collect time and/or memory
  statistics. They may generate output suitable for graphing. They are
  run on demand, not by automated tools. They are enclosed by
  \ccode{\#ifdef eslFOO\_BENCHMARK} tags.

\item[\textbf{Statistics generators.}] (Optional.) These tests collect
  statistics used to characterize the module's scientific performance,
  such as its accuracy at some task. They may generate graphing
  output. They are run on demand, not by automated tools. They are
  enclosed by \ccode{\#ifdef eslFOO\_STATS} tags.

\item[\textbf{Examples.}] Every module has at least one example
  \ccode{main()} that provides a ``hello world'' level example of
  using the module's API. Examples are enclosed in \ccode{cexcerpt}
  tags for extraction and verbatim inclusion in the documentation.
  They are enclosed by \ccode{\#ifdef eslFOO\_EXAMPLE} tags, where
  \ccode{FOO} is the name of the module.
\end{description}  

All modules have at least one test driver and one example. Other tests
and examples are optional. When there is more than one \ccode{main()}
of a given type, the additional tags are numbered starting from 2: for
example, a module with three example \ccode{main()'s} would have three
tags for conditional compilation, \ccode{eslFOO\_EXAMPLE},
\ccode{eslFOO\_EXAMPLE2}, and \ccode{eslFOO\_EXAMPLE3}.

These drivers appear as separate sections in that order. The test
driver is first so it's close to the unit test code. The examples are
last so they are easy to locate.

The format of the conditional compilation tags for all the drivers
(including test and example drivers) must be obeyed. Some test scripts
are scanning the .c files and identifying these tags
automatically. For instance, the driver compilation test identify any
tag named
\ccode{esl\$(MODULENAME)\_\{TESTDRIVE,EXAMPLE,REGRESSION,BENCHMARK,STATS\}*}
and attempt to compile the code with that tag defined.

Which driver is compiled (if any) is controlled by conditional
compilation of the module's \ccode{.c} file with the appropriate
tag. For example, to compile and run the \eslmod{sqio} test driver as
a standalone module:

\begin{cchunk}
   %  gcc -g -Wall -I. -o test -DeslSQIO_TESTDRIVE esl_sqio.c easel.c -lm
   %  ./test
\end{cchunk}

or to compile and run it in full library configuration:

\begin{cchunk}
   %  gcc -g -Wall -I. -L. -o test -DeslSQIO_TESTDRIVE esl_sqio.c -leasel -lm
   %  ./test
\end{cchunk}

\subsection{The test driver}

One of the more important components of an Easel module is the
automated test driver. This \ccode{main()} is used in unit testing,
memory checking, and code coverage analysis.

A test driver shall work with the module in minimal standalone configuration, in any
possible augmented configuration, or in full library
configuration. Example compilations:
\begin{cchunk}
      % gcc -g -Wall -o test -I. -DeslMSA_TESTDRIVE esl_msa.c easel.c -lm
      % gcc -g -Wall -o test -I. -DeslMSA_TESTDRIVE -DeslAUGMENT_ALPHABET esl_msa.c esl_alphabet.c easel.c -lm
      % gcc -g -Wall -o test -I. -L. -DeslMSA_TESTDRIVE esl_msa.c -leasel -lm
\end{cchunk}

A test driver shall require no command line arguments, and shall
require no external files, except for tmpfiles that it generates.

Test drivers should not use command line options at all, because of
the requirement that the driver must compile in a minimal standalone
configuration (i.e., without \eslmod{getopts}). 

A test driver returns zero (\ccode{eslOK}) upon success, and a nonzero
code on failure, as is standard for UNIX command line
applications. Automated test harnesses (including \ccode{sqc}) rely on
this.

A test driver generates no output on \ccode{stdout} by default. 

Like unit tests, a test driver (and any tests it runs) shall terminate
execution as soon as possible to facilitate tracing/debugging by
calling \ccode{esl\_fatal()}, with an informative error message. 

\footnote{Calling \ccode{abort()} doesn't work because it generates an
abnormal signal that messes up \ccode{sqc}'s output. Calling
\ccode{exit(1)} doesn't leave an informative message, so a command
line user can't readily tell an error occurred.  Throwing an exception
doesn't work because we will also be testing the code under a nonfatal
exception handler.}  Garbage collection upon a failure (including
memory, opened files, or created temporary files) is not required, and
may even be counterproductive.  For example, for debugging purposes,
it may be advantageous not to delete a temporary file from a failed
test.

Thus, from the command line, a successful test run looks like:

\begin{cchunk}
      % ./test
      %
\end{cchunk}

and a failure looks like:

\begin{cchunk}
      % ./test
      yoyodyne exercise on fire, send help.
      %
\end{cchunk}




\subsubsection{using temporary files in test drivers}

If a unit test or testdriver needs to create a named temporary file
(to test i/o), the tmpfile is created with
\ccode{esl\_tmpfile\_named()}:

\begin{cchunk}
   char  tmpfile[16] = "esltmpXXXXXX";
   FILE *fp;

   if (esl_tmpfile_named(tmpfile, &fp) != eslOK) esl_fatal("failed to create tmpfile");
   write_stuff_to(fp);
   fclose(fp);

   if ((fp = fopen(tmpfile)) == NULL) esl_fatal("failed to open tmpfile");
   read_stuff_from(fp);
   fclose(fp);

   remove(tmpfile);
\end{cchunk}

Thus tmp files created by Easel's test suite have a common naming
convention, and are put in the current working directory. On a test
failure, the tmp file remains, to assist debugging; on a test success,
the tmp file is removed. The \ccode{make clean} targets in Makefiles
are looking to remove files matching the target \ccode{esltmp??????}.






%%%%%%%%%%%%%%%%%%%%%%%%%%%%%%%%%%%%%%%%%%%%%%%%%%%%%%%%%%%%%%%%
\section{autodoc - extracting LaTeX documentation from source files}
%%%%%%%%%%%%%%%%%%%%%%%%%%%%%%%%%%%%%%%%%%%%%%%%%%%%%%%%%%%%%%%%





%%%%%%%%%%%%%%%%%%%%%%%%%%%%%%%%%%%%%%%%%%%%%%%%%%%%%%%%%%%%%%%%
\section{cexcerpt - extracting C source snippets}
%%%%%%%%%%%%%%%%%%%%%%%%%%%%%%%%%%%%%%%%%%%%%%%%%%%%%%%%%%%%%%%%

This book includes many examples of C code that are extracted verbatim
from Easel's C source files, using \prog{cexcerpt}, which is part of
the SSDK.

The \ccode{documentation/Makefile} runs \prog{cexcerpt} on every
module .c and .h file. The extracted cexcerpts are placed in .tex
files in the temporary \ccode{cexcerpts/} subdirectory.

Usage: \ccode{cexcerpt <file.c> <dir>}. Processes C source file
\ccode{file.c}; extracts all tagged excerpts, and puts them in a file
in directory \ccode{<dir>}.

An excerpt is marked with special comments in the C file:
\begin{cchunk}
/*::cexcerpt::my_example::begin::*/
   while (esl_sq_Read(sqfp, sq) == eslOK)
     { n++; }
/*::cexcerpt::my_example::end::*/
\end{cchunk}

The cexcerpt marker's format is \ccode{::cexcerpt::<tag>::begin::} (or
end). A comment containing a cexcerpt marker must be the first text on
the source line. A cexcerpt comment may be followed on the line by
whitespace or a second comment.

The \ccode{<tag>} is used to construct the file name, as
\ccode{<tag>.tex}.  In the example, the tag \ccode{my\_example} creates
a file \ccode{my\_example.tex} in \ccode{<dir>}.

All the text between the cexcerpt markers is put in the file.  In
addition, this text is wrapped in a \ccode{cchunk} environment.  This
file can then be included in a \LaTeX\ file.

For best results, the C source should be free of TAB characters.
"M-x untabify" on the region to clean them out.

Cexcerpts can't overlap or nest in any way in the C file. Only one tag
can be active at a time.

\section{Automated test reports}

  \subsection{Unit tests}

  \subsection{Driver compilation checks}

  \subsection{Code coverage (gcov)}

  \subsection{Memory checking (valgrind)}



%%%%%%%%%%%%%%%%%%%%%%%%%%%%%%%%%%%%%%%%%%%%%%%%%%%%%%%%%%%%%%%%
\section{Development environment}
%%%%%%%%%%%%%%%%%%%%%%%%%%%%%%%%%%%%%%%%%%%%%%%%%%%%%%%%%%%%%%%%

Easel is developed on GNU/Linux systems with the following freely
available tools installed:

{\small
\begin{tabular}{ll}
GNU gcc         & Compiler \\
GNU gdb         & Debugger\\
GNU autoconf    & Platform-independent Makefile generator\\
GNU make        & Make utility\\
GNU emacs       & Editor    \\
GNU gprof       & Profiling and optimization tool \\
GNU gcov        & Code coverage analysis\\
Perl            & Scripting language\\
LaTeX           & Typesetting\\
Subversion      & Revision control\\
Valgrind        & Memory bounds and leak checking\\
\end{tabular}
}



