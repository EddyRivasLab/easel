
The getopts module is a more heavyweight version of command line
parsing functions like POSIX \ccode{getopt()} and GNU's
\ccode{getopt\_long()}. The familiar syntax of command line options is
the same (both standard POSIX one-character options and GNU long
options), but much of the tedious checking of the validity of option
and command line arguments that robust C applications are normally
obligated to do for themselves is handled automatically by the
internals of the Easel getopts module.

The Easel getopts API allows options to be set from the command line,
environment variables, and one or more configuration files.  Option
arguments and command line arguments may be automatically type checked
(for instance, making sure an integer is provided when one is
expected) and range checked (for instance, making sure that an
argument that must be a probability in the range of 0.0-1.0 really
is). Options may be linked into ``toggle groups'', such that setting
one option automatically unsets others. One can specify that an option
makes no sense unless other required options are also set, or
conversely that an option is incompatible with one or more other
options being set. The entire configuration state of the application
is stored in a single object.

\subsection{The getopts API}

The module implements a \ccode{ESL\_GETOPTS} object that holds the
configuration state of the application, and an \ccode{ESL\_OPTIONS}
structure that contains information about one configurable option.  An
application defines an array of \ccode{ESL\_OPTIONS} to declare what
options it will allow.

The API defines the following functions:

\begin{tabular}{ll}
       \multicolumn{2}{c}{\textbf{creating/destroying a configuration}}\\
\ccode{esl\_getopts\_Create()}    & Creates a new \ccode{ESL\_GETOPTS} object. \\
\ccode{esl\_getopts\_Destroy()}   & Destroys a created \ccode{ESL\_GETOPTS} object. \\
       \multicolumn{2}{c}{\textbf{setting a configuration}}\\
\ccode{esl\_opt\_ProcessConfigFile()}  & Parses options out of a configuration file.\\
\ccode{esl\_opt\_ProcessEnvironment()} & Parses options out of environment variables.\\
\ccode{esl\_opt\_ProcessCmdline()}     & Parses options from the command line.\\
       \multicolumn{2}{c}{\textbf{final verification of a configuration}}\\
\ccode{esl\_opt\_VerifyConfig()}       & Tests required and incompatible opts.\\
       \multicolumn{2}{c}{\textbf{retrieving option settings}}\\
\ccode{esl\_opt\_GetBooleanOption()}   & Retrieve the state of a TRUE/FALSE option setting.\\
\ccode{esl\_opt\_GetIntegerOption()}   & Retrieve an integer option argument.\\
\ccode{esl\_opt\_GetFloatOption()}     & Retrieve a float option argument.\\
\ccode{esl\_opt\_GetDoubleOption()}    & Retrieve a double option argument.\\
\ccode{esl\_opt\_GetCharOption()}      & Retrieve a single-character option argument.\\
\ccode{esl\_opt\_GetStringOption()}    & Retrieve an option argument as a string.\\
       \multicolumn{2}{c}{\textbf{retrieving command line arguments}}\\
\ccode{esl\_opt\_GetCmdlineArg()}      & Retrieve the next command line argument.\\
\end{tabular}

\subsection{Examples of using the getopts API}

\subsection{Features of getopts}

   \subsubsection{Type checking}
   
   \subsubsection{Range checking}

   \subsubsection{Default values}

   \subsubsection{Setting toggle groups of options}

   \subsubsection{Specifying required or incompatible options}

The \ccode{required\_opts} and \ccode{incompat\_opts} fields
SRE STOPPED HERE.
 of the \ccode{ESL\_OPTIONS} structure
specified a comma-delimited list of options that may not also be on if
this option is on. For example, \ccode{--Set this field to NULL if no
such checking is necessary.

   \subsubsection{Specifying incompatible options}

The \ccode{incompat\_opts} field of the \ccode{ESL\_OPTIONS} structure
specified a comma-delimited list of options that may not also be on if
this option is on. For example, \ccode{--Set this field to NULL if no such checking is
necessary.

\subsection{Application configuration using getopts}

   \subsubsection{Command line option syntax}

Command line option syntax is standard, essentially identical to the
syntax used by GNU programs.

Options are either short or long. Short options are a single character
preceded by a single \ccode{-}; for example, \ccode{-a}. Long options
are preceded by two dashes, and can have any wordlength; for example,
\ccode{--option1}.

If a short option takes an argument, the argument may either be
attached (immediately follows the option character) or unattached (a
space between the optchar and the argument. For example, \ccode{-n5}
and \ccode{-n 5} both specify an argument \ccode{5} to option
\ccode{-n}.

Short options can be concatenated into a string of characters;
\ccode{-abc} is equivalent to \ccode{-a -b -c}. (Concatenation may
only be used on the command line, not in configuration files or in
fields of the \ccode{ESL\_OPTIONS} structure array.) Only the last
option in such a string can take an argument, and the other options in
the optstring must be simple on/off booleans. For example, if
\ccode{-a} and \ccode{-b} are boolean switches, and \ccode{-W} takes a
\ccode{<string>} argument, either \ccode{-abW foo} or \ccode{-abWfoo}
is correct, but \ccode{-aWb foo} is not.

For a long option that takes an argument, the argument can be provided
either by \ccode{--foo arg} or \ccode{--foo=arg}.

Long options may be abbreviated, if the abbreviation is unambiguous;
for instance, \ccode{--foo} or \ccode{--foob} suffice to active an
option \ccode{--foobar}. (Like concatenation of short options,
abbreviation of long options is a shorthand that may only be used on
the command line.)

Long option names should contain only alphanumeric characters,
\ccode{-}, or \ccode{\_}. They may not contain \ccode{=} or \ccode{,}
characters, which would confuse the option argument parsers. Other
characters should work but are not recommended.

Multi-word arguments may be quoted: for example, \ccode{--hostname "my
host"}.

   \subsubsection{Configuration file format}

Each line of a configuration file contains an option and an argument
(if the option takes an argument). Blank lines are ignored.  Anything
following a \ccode{\#} character on a line is a comment and is
ignored. The syntax of options and arguments is stricter than on
command lines.  Concatenation of short options is not allowed,
abbreviation of long options is not allowed, and arguments must always
be separated from options by whitespace. For example:

\begin{cchunk}
   # Customized configuration file for my application.
   #
   -a             # Turn -a on.
   -b             # Turn -b on.
   -W arg         # Set -W to "arg"
\end{cchunk}

   \subsubsection{Environment variables}

When a non-NULL environment variable name is connected to an option,
\ccode{esl\_opt\_ProcessEnvironment()} looks for that name in the
environment and sets the option value accordingly. Boolean options are
set by setting the environment variable with no argument, for
instance (in a C-shell),

\begin{cchunk}
  % setenv FOO_DEBUGGING
\end{cchunk}

and other options are set by setting the envvar to the appropriate
argument, for instance (in a C-shell),

\begin{cchunk}
  % setenv FOO_DEBUG_LEVEL 5
\end{cchunk}




