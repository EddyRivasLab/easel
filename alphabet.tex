The alphabet (abc) module contains routines for digitizing alphabetic
biosequences.

For many purposes, it is more convenient to represent the nucleotides
and amino acids as array indices 0..3 or 0..19, respectively. It is
also convenient to index a sequence as 1..L coordinates instead of C's
0..L-1 array representation; in part for readability, and also because
some codes (DP alignment algorithms, for example) have boundaries
where coordinate 0 is reserved for an initialization condition.

The digital alphabet accommodates gap symbols and IUPAC/IUBMB
degenerate residue nomenclature. It allows for "synonymous" letter
codes to map to the same residue (such as allowing T/U to mean the
same thing in nucleic acid sequences, allowing technically non-IUBMB
codes like the use of X to mean N in DNA sequences, or mapping
modified residues onto the basic alphabet, such as mapping U
(selenocysteine) onto S (serine) or X (unknown) for different
purposes. Easel maintains all of this information in an
\ccode{ESL\_ALPHABET} structure, which an application usually creates
once, typically (but not necessarily) as a global variable.

\subsection{API}

A standard biosequence alphabet is created using
\ccode{esl\_alphabet\_Create(type)}, where \ccode{type} can be
\ccode{eslDNA}, \ccode{eslRNA}, or \ccode{eslAMINO}.

An input sequence \ccode{seq} of length \ccode{L} is digitized
according to alphabet \ccode{a}, creating a newly allocated digital
sequence \ccode{dsq}, by calling
\ccode{esl\_abc\_CreateDigitalSequence(a, seq, L, \&dsq)}. The caller
must free \ccode{dsq} using \ccode{free(dsq)}.  Alternatively, if the
caller has already allocated \ccode{L+2} (or more) bytes in
\ccode{dsq}, it can call \ccode{esl\_abc\_DigitizeSequence(a, seq, L,
dsq)}, which is the non-allocating version of
\ccode{esl\_abc\_CreateDigitalSequence()}. 

For an input sequence of length L, the digitized sequence (dsq) is an
unsigned char array of L+2 bytes. \ccode{dsq[0]} and \ccode{dsq[L+1]}
contain a sentinel byte of value \ccode{eslSENTINEL} (255).  Positions
1..L hold the residues, where values 0..3 encode ACGT in DNA
sequences, 0..3 encode ACGU in RNA sequences, and 0..19 encode AC..WY
in amino acid sequences.

Both sequence-digitizing functions return \ccode{ESL\_EINVAL} if the
input sequence contains characters that are not in the
alphabet. Because input sequences are often provided by a user (not
the program), this is a common error that the application must check
for.

An example of creating a standard alphabet and digitizing a sequence
is:

\begin{cchunk}
  ESL_ALPHABET  *a;
  char           dnaseq[] = "GARYTC";
  int            L        = 6;
  unsigned char *dsq;
  
  a = esl_alphabet_Create(eslDNA);

  if (esl_abc_CreateDigitalSequence(a, dnaseq, L, &dsq) == ESL_EINVAL)
   {
     fprintf(stderr, "Invalid character in the input sequence");
     exit(1);
   }

  free(dsq);
  esl_alphabet_Destroy(a);
  return 0;
}
\end{cchunk}

For more working examples of the API, see the test driver at the end
of \cfile{alphabet.c}.

\subsection{Standard alphabets: DNA, RNA, protein}

Easel's \emph{internal alphabet} is a string (\ccode{a->sym}) of
length K', which contains the K symbols of the base alphabet, a gap
character, a degenerate ``any'' character, and (optionally) additional
degenerate residue codes. The digital index used for each residue is
the index of a residue in this string, 0..K'-1.

The three standard internal alphabets are:

\begin{tabular}{lllrr}
\textbf{Type} & \textbf{sym}                     & Synonyms &\textbf{K} & \textbf{K'} \\
eslDNA        & \ccode{ACGT-RYMKSWHBVDN}         & U=T; X=N & 4         &  16         \\
eslRNA        & \ccode{ACGU-RYMKSWHBVDN}         & T=U; X=N & 4         &  16         \\
eslAMINO      & \ccode{ACDEFGHIKLMNPQRSTVWY-BZX} & U=S      & 20        &  24         \\
\end{tabular}

The first K residues ([0..K-1]) are the \emph{base alphabet} (``ACGT''
for example); the next residue (K) is always a gap symbol ('-' for
example); the last residue (K'-1) is the degenerate code for ``any''
or ``unknown'' ('N', for example); and the remaining residues
([K+1..K'-2]) are the \emph{degenerate alphabet}, implementing
standard IUPAC or IUBMB one-letter codes.

The \ccode{sym} string is used to translate digital sequences back to
output sequences for display, so it may be referred to as the internal
alphabet or the output alphabet.  Thus, only the residues in the
\ccode{sym} can be represented in output from Easel, although an
application may convert some characters for its own purposes before
displaying an alphabetic string (to use different gap symbols for
insertions versus deletions, for instance; or to use upper/lower case
conventions to represent match/insert positions).

The DNA and RNA alphabets follow published IUBMB recommendations
("Nomenclature for incompletely specified bases in nucleic acid
sequences", Eur. J. Biochem. 150:1-5, 1985;
\url{http://www.chem.qmul.ac.uk/iubmb/misc/naseq.html}), with the
addition of X as a synonym for N (acquiescing to the BLAST filter
standard of using X's to mask residues), and the use of U in RNA
sequences in place of T; the use of ``synonyms'' is described in the
following section on the input map.

The one-letter code for amino acids follows section 3AA-21 of the
IUPAC recommendations ("Nomenclature and symbolism for amino acids and
peptides", Eur. J. Biochem. 138:9-37, 1985;
\url{http://www.chem.qmul.ac.uk/iupac/AminoAcid/}]; augmented by U for
selenocysteine, as recommended by the JCBN/NC-IUBMB Newsletter, 1999
(\url{http://www.chem.qmul.ac.uk/iubmb/newsletter/1999/item3.html}),
though it is not really a "degenerate" residue. Since we must map it
onto one of the 20-letter code, we map it as a synonym of serine (S).

\subsection{The input map}

Input symbol sequences are converted into the internal digitized
alphabet. This is facilitated by an \emph{input map}, which translates
ASCII codes to the appropriate digitization. Symbols in the internal
alphabet are mapped in the obvious way ('A' $\rightarrow$ 0, etc.);
the input map additionally allows \emph{synonymous} input symbols,
such as allowing T or U to be equivalent in DNA/RNA sequences, or
allowing upper or lower case residues to be equivalent, or allowing
additional gap symbols other than the standard one.

Individual synonyms are added to an alphabet's input map by the call
\ccode{esl\_alphabet\_SetSynonym()}; for example,
\ccode{esl\_alphabet\_SetSynonym(alphabet, 'U', 'S')} indicates that a
'U' (selenocysteine) in a sequence should be mapped onto 'S' (serine)
internally in \ccode{alphabet}.

\ccode{esl\_alphabet\_SetCaseInsensitive()} makes both upper case and
lower case input alphabetic characters map to their equivalent in the
internal alphabet in a case-insensitive manner.  This function works
only on residues that have already been declared to be part of the
alphabet, so it must be called \emph{after} all synonyms have been
added.

The standard DNA alphabet is set to be case-insensitive for input, and
it defines U as a synonym for T (to allow input of RNA sequences as
DNA), X as a synonym for N (because many sequence masking programs
mistakenly use a non-IUBMB degeneracy symbol X to mean N, and we have
to live with it), and '\_' and '.' as gap symbols in addition to
'-'. The standard RNA alphabet is the same (except that it defines T
as a synonym for U). (See \ccode{alphabet.c:create\_dna()}.)

The standard protein alphabet is set to be case-insensitive for input.
It defines U (selenocysteine) as a synonym for S (serine); not because
this is biologically correct, but solely so that rare
selenocysteine-containing sequences can be accepted as input and
treated somewhat sensibly. And, like the nucleic acid alphabets, it
adds '\_' and '.' as gap symbols in addition to '-'. (See
\ccode{alphabet.c:create\_amino()}.)

\subsubsection{Implementation of the input map}

The \emph{input map} is implemented as an array
\ccode{a->inmap[0..127]}.  \ccode{x = a->inmap[sym]} is the digital
representation of \ccode{sym}, where $x = 0..K'-1$ or
\ccode{eslSENTINEL}, and \ccode{sym} is a 7-bit unsigned ASCII
character in the range 0..127.  The flag \ccode{a->inmap[sym] =
eslSENTINEL} indicates that the symbol is unrecognized (not part of
the legal alphabet).  The internal alphabet is mapped
\ccode{a->inmap[a->sym[i]] = i} when an alphabet is created.


\subsection{Degenerate residue codes}

Dealing with real biosequences means dealing with some rarely
occurring degenerate residue codes. Easel implements the standard
IUPAC and IUBMB one-letter degeneracy codes for protein and nucleic
acid, and provides an API to simplify an application's scoring of
degenerate residue codes.

When creating an alphabet, each degenerate symbol in the internal
alphabet (residues \ccode{K+1}..\ccode{Kp-2}) must be initialized by
calling \ccode{esl\_alphabet\_SetDegeneracy(alphabet, c, syms)}, which
assigns degenerate alphabetic symbol \ccode{c} to the alphabetic
symbols in the string \ccode{syms}; for example,
\ccode{esl\_alphabet\_SetDegeneracy(a, 'R', ``AG'')} assigns R
(purine) to mean A or G.  For the standard biosequence alphabets, this
is done automatically to define the IUPAC degeneracy codes for amino
acid residues and the IUBMB degeneracy codes for nucleic acid
residues.  (The last residue, K'-1, is always the ``any'' character in
any alphabet, and is initialized automatically in any alphabet, even
customized alphabets.)

To score a degenerate residue code $x$, Easel provides two kinds of
functions. One set of functions assigns an average score:

\[
  S(x) =  \frac{\sum_{y \in D(x)}  S(y) } { |D(x)| },
\]

where $D(x)$ is the set of residues $y$ represented by degeneracy code
$x$ (for example, $D(\mbox{R}) = \{ \mbox{A,G} \}$), $| D(x) |$ is the
number of residues that the degeneracy code includes, and $S(y)$ is
the score of a base residue $y$. Because scores $S(y)$ are commonly
kept as integers, floats, or doubles, three functions are provided
that differ only in the storage type of the scores:
\ccode{esl\_abc\_AvgIScore(a,x,sc)}, \ccode{esl\_abc\_AvgFScore(a,x,sc)}, and
\ccode{esl\_abc\_AvgDScore(a,x,sc)}, respectively, calculate an average
score of residue \ccode{x} in alphabet \ccode{a} given a base score
vector \ccode{sc[0]..sc[K-1]}.

A second set of functions assigns an expected score, weighted by an
expected occurrence probability $p_y$ of the residues $y$ (often the
random background frequencies):

\[
  S(x) =  \frac{\sum_{y \in D(x)}  p_y S(y) } { \sum_{y \in D(x)} p_y },
\]

These three functions are \ccode{esl\_abc\_ExpectIScore(a,x,sc,p)},
\ccode{esl\_abc\_ExpectFScore(a,x,sc,p)}, and
\ccode{esl\_abc\_ExpectDScore(a,x,sc,p)}, respectively. For the integer
and float versions, the probability vector is in floats, and for the
double version, the probability vector is in doubles.

An application would typically use these functions in one of two
strategies to score degenerate codes: precalculation, or on-the-fly.
To precalculate (as in HMMER), position-specific score vectors are
length K', inclusive of degeneracy codes; the degenerate parts of
these score vectors (positions \ccode{K+1..Kp-1}) are initialized
using the appropriate function. 

To score residues on-the-fly using base score vectors of length
\ccode{K}, each input residue \ccode{x} is tested before scoring it
appropriately.  \ccode{esl\_abc\_IsBasic(a, x)} returns \ccode{TRUE} if
\ccode{x} is in the basic set of \ccode{K} residues in alphabet
\ccode{a}, and \ccode{FALSE} otherwise. Similarly,
\ccode{esl\_abc\_IsGap(a,x)} tests whether $x$ is a gap, and
\ccode{esl\_abc\_IsDegenerate(a,x)} tests whether $x$ is a degenerate
residue.

\subsubsection{Implementation of the degeneracy map}

The degeneracy map is implemented in a $K' \times K$ array
\ccode{a->degen[0..Kp-1][0..K-1]} of 1/0 (TRUE/FALSE) flags, where
\ccode{a->degen[x][y] == TRUE} indicates that the residue set $D(x)$
for degeneracy code \ccode{x} contains basic residue \ccode{y}.
\ccode{a->ndegen[x]} contains the cardinality $|D(x)|$ of the residue
set for degeneracy code \ccode{x}.

The degeneracy map is undefined for the special case of $x=K$, the
digitized gap symbol.

The last character in the internal alphabet is automatically assumed
to be an ``any'' character (such as 'N' for DNA or RNA, 'X' for
protein); \ccode{a->degen[Kp-1][i] = 1} for $i=0..K-1$, and
\ccode{a->ndegen[Kp-1] = K} when the alphabet is created. All other
degeneracy codes are added when the alphabet is created using
\ccode{esl\_alphabet\_SetDegeneracy(alphabet, c, syms)}.

\subsection{Creating custom alphabets}

An example of creating a customized alphabet follows.

\begin{cchunk}
  ESL_ALPHABET *a;

  /* 1. Create the base alphabet structure.
   */
  a = esl_alphabet_CreateCustom("ACDEFGHIKLMNPQRSTVWY-BZX", 20, 24);

  /* 2. Set synonyms in the input map, by calls to
   *    esl_alphabet_SetSynonym.
   */
  esl_alphabet_SetSynonym(a, 'U', 'S');     /* read selenocysteine U as a serine S */
  esl_alphabet_SetSynonym(a, '.', '-');     /* allow . as a gap character too */

  /* 3. After all synonyms are set, (optionally) make
   *    the map case-insensitive.
   */
  esl_alphabet_SetCaseInsensitive(a);       /* allow lower case input */

  /* 4. Define all the degeneracy codes in the alphabet.
   */
  esl_alphabet_SetDegeneracy(a, 'Z', "QE"); /* read Z as {Q|E} */

  /* (use the alphabet to read/digitize/score sequences */

  esl_alphabet_Destroy(a);
\end{cchunk}



