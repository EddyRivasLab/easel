
The vectorops (vec) module contains small but useful operations on
vectors. 

Different functions allow an operation to be performed in vectors
containing elements of different scalar types (double, float,
integer). The appropriate routine is prefixed with D, F, or I. For
example, \ccode{esl\_vec\_DSet()} is the Set routine for a vector of
doubles; \ccode{esl\_vec\_ISet()} is for integers.

\subsection{The vectorops API}

\begin{tabular}{ll}
   \multicolumn{2}{c}{\textbf{All vectors}}\\
\ccode{esl\_vec\_\{D,F,I\}Set()}         & Set all items in vector to scalar value.\\
\ccode{esl\_vec\_\{D,F,I\}Scale()}       & Multiply all items in vector by scalar.\\
\ccode{esl\_vec\_\{D,F,I\}Increment()}   & Add a scalar to all items in vector.\\
\ccode{esl\_vec\_\{D,F,I\}Sum()}         & Return scalar sum of values in vector.\\
\ccode{esl\_vec\_\{D,F,I\}Add()}         & Add vec2 to vec1.\\
\ccode{esl\_vec\_\{D,F,I\}Copy()}        & Set vec1 to be same as vec2. \\
\ccode{esl\_vec\_\{D,F,I\}Dot()}         & Return dot product of two vectors.\\
\ccode{esl\_vec\_\{D,F,I\}Max()}         & Return value of maximum element in vector.\\
\ccode{esl\_vec\_\{D,F,I\}Min()}         & Return value of minimum element in vector.\\
\ccode{esl\_vec\_\{D,F,I\}ArgMax()}      & Return index of maximum element in vector.\\
\ccode{esl\_vec\_\{D,F,I\}ArgMin()}      & Return index of minimum element in vector.\\
    \multicolumn{2}{c}{\textbf{Probability vectors}}\\
\ccode{esl\_vec\_\{D,F\}Norm()}          & Normalize a probability vector of length n.\\
\ccode{esl\_vec\_\{D,F\}Log()}           & Convert all items in vec to log probabilities. \\
\ccode{esl\_vec\_\{D,F\}Entropy()}       & Return Shannon entropy of probability vector, in bits\\
    \multicolumn{2}{c}{\textbf{Log probability vectors}}\\
\ccode{esl\_vec\_\{D,F\}Exp()}           & Convert log p's back to probabilities\\
\ccode{esl\_vec\_\{D,F\}LogSum()}        & Given vector of log p's; return log of summed p's.\\
\ccode{esl\_vec\_\{D,F\}LogNorm()}       & Normalize a log p vector, making it a prob vector. \\
\end{tabular}


