
The getopts module interprets the familiar UNIX command line syntax,
allowing both standard POSIX one-character options and GNU-style long
options, in addition to command line arguments. The implementation
shares similarities with POSIX \ccode{getopt()} and GNU's
\ccode{getopt\_long()}, but has somewhat more power, at the cost of
enforcing a fairly specific style.

Options can be set from the command line, environment variables, and
one or more configuration files.

Option and commandline arguments can be automatically checked for
valid type (integers, real numbers, characters, or strings), and
numeric arguments can be checked for valid range (for instance,
ensuring that a probability is in the range $0 \leq x \leq 1$).

Options can be linked into ``toggle groups'', such that setting one
option automatically unsets others. 

You can specify that an option makes no sense unless other required
options are also set, or conversely that an option is incompatible
with one or more other options being set. 

A standardized usage display for command line options can be printed
directly from the internal information, including default values and
range restrictions when line length allows. 

You configure all this by defining an array of \ccode{ESL\_OPTIONS}
structures that provide the necessary information. This array is
passed into a \ccode{ESL\_GETOPTS} object, which is used to determine
and store the configuration state of your application according to the
command line, environment variables, and configuration files.

\subsection{The getopts API}

The module implements a \ccode{ESL\_GETOPTS} object that holds the
configuration state of the application, and an \ccode{ESL\_OPTIONS}
structure that contains information about one configurable option.  An
application defines an array of \ccode{ESL\_OPTIONS} to declare what
options it will allow.

The API defines the following functions:

\begin{tabular}{ll}
       \multicolumn{2}{c}{\textbf{creating/destroying a configuration}}\\
\ccode{esl\_getopts\_Create()}    & Creates a new \ccode{ESL\_GETOPTS} object. \\
\ccode{esl\_getopts\_Destroy()}   & Destroys a created \ccode{ESL\_GETOPTS} object. \\
\ccode{esl\_getopts\_Dump()}      & Dumps a configuration to an output stream. \\
       \multicolumn{2}{c}{\textbf{setting a configuration}}\\
\ccode{esl\_opt\_ProcessConfigFile()}  & Parses options out of a configuration file.\\
\ccode{esl\_opt\_ProcessEnvironment()} & Parses options out of environment variables.\\
\ccode{esl\_opt\_ProcessCmdline()}     & Parses options from the command line.\\
       \multicolumn{2}{c}{\textbf{final verification of a configuration}}\\
\ccode{esl\_opt\_VerifyConfig()}       & Tests required and incompatible opts.\\
       \multicolumn{2}{c}{\textbf{retrieving option settings}}\\
\ccode{esl\_opt\_GetBooleanOption()}   & Retrieve the state of a TRUE/FALSE option setting.\\
\ccode{esl\_opt\_GetIntegerOption()}   & Retrieve an integer option argument.\\
\ccode{esl\_opt\_GetFloatOption()}     & Retrieve a float option argument.\\
\ccode{esl\_opt\_GetDoubleOption()}    & Retrieve a double option argument.\\
\ccode{esl\_opt\_GetCharOption()}      & Retrieve a single-character option argument.\\
\ccode{esl\_opt\_GetStringOption()}    & Retrieve an option argument as a string.\\
       \multicolumn{2}{c}{\textbf{retrieving command line arguments}}\\
\ccode{esl\_opt\_ArgNumber()}          & Retrieve number of cmdline args.\\
\ccode{esl\_opt\_GetCmdlineArg()}      & Retrieve the next command line argument.\\
\end{tabular}

\subsection{An example of using the getopts API}

The steps in using the getopts API are:

\begin{itemize}
\item The application defines an array of \ccode{ESL\_OPTIONS}
      structures, one per option. Name, type, and default fields are
      required; the other fields are optional.  The array is
      terminated by an entry of all 0's.

\item The application defines a helpful ``usage'' string. If the getopts
      module encounters any problem with the configuration, it prints
      an error message followed by this usage string.

\item A \ccode{ESL\_GETOPTS} object is created, using the options
      array and the usage string. At this point, all options are 
      initialized to default values inside the object.

\item The application now processes option settings from the command
      line, environment variables, and one or more configuration
      files. 
 
\item The application recovers option settings one by one, and stores
      them however and wherever it wants. 

\item Command line arguments are then recovered, in order.

\item Once the application has recovered all of its options and
      arguments, it may free the \ccode{ESL\_GETOPTS} object. This 
      object is the only place where memory is allocated; any string
      retrieved as an option or argument, for example, is a pointer
      to internal memory maintained by the object.
\end{itemize}

For example, the code below sets up four short options and two long
options, without using any of getopts' optional validation or
configuration mechanisms:

\begin{figure}
\input{cexcerpts/getopts_example1}
\caption{An example of using the getopts module.}
\label{fig:getopts_example1}
\end{figure}

An example of running this program:
\begin{cchunk}
   % ./example1 -ax 0.3 -n 42 --file foo --char x baz
   Option -a:      on
   Option -b:      on
   Option -n:      42
   Option -x:      0.300000
   Option --file:  foo
   Option --char:  x
   Cmdline arg:    baz
\end{cchunk}

Note that because we set the default value of \ccode{-b} to TRUE in
this example, it is always on whether we use the \ccode{-b} option or
not.


\subsection{Defining options in the \ccode{ESL\_OPTIONS} array}

The \ccode{ESL\_OPTIONS} structure is declared in \ccode{getopts.h}
as:

\input{cexcerpts/options_object}

Aside from the option name, which is self-explanatory, how to set each of these
fields in an options array is described in detail below:

   \subsubsection{Type checking}

Five argument types are recognized:

\begin{center}
\begin{tabular}{lll}
\textbf{flag}           & \textbf{description}    & \textbf{type checking} \\\hline
\ccode{eslARG\_NONE}     & Boolean switch (on/off) & n/a                   \\
\ccode{eslARG\_INT}      & integer                 & convertible by \ccode{atoi()}\\
\ccode{eslARG\_REAL}     & float or double         & convertible by \ccode{atof()}\\
\ccode{eslARG\_CHAR}     & one character           & single ASCII char \\
\ccode{eslARG\_STRING}   & any string              & not checked\\
\end{tabular}
\end{center}

All arguments are declared, configured, and stored internally as
strings in a \ccode{ESL\_GETOPTS} object. For arguments that are
declared to be of types \ccode{eslARG\_INT}, \ccode{eslARG\_REAL}, or
\ccode{eslARG\_CHAR}, the string is checked to be sure it can be
completely converted to the declared type.

Strings (filenames, whatever) are of type \ccode{eslARG\_STRING}, and
since any string is valid (including a NULL pointer), this type is not
checked. An application can also declare an argument to be of type
\ccode{eslARG\_STRING} if for some reason it wants to bypass type
checking. The application would recover the option setting with
\ccode{esl\_opt\_GetStringOption()} and then deal with any type
conversion itself.

   \subsubsection{Default values}

Since the \ccode{ESL\_GETOPTS} object stores all values internally as
strings, default settings in the options array are also all provided
as strings.

Boolean defaults should be provided as \ccode{FALSE} or
\ccode{\"TRUE\"}.  The syntax here is a little weird; the quotes
around TRUE and the absence of quotes around FALSE are
important. Internally, booleans are off if they're NULL, and on if
they're non-NULL. Therefore default values of \ccode{FALSE},
\ccode{NULL}, or \ccode{0} are all interpreted as ``off'', and any
non-NULL string such as \ccode{"TRUE"}, \ccode{"on"}, or even
\ccode{""} is interpreted as ``on''.

Integer, real-valued, and character arguments are provided as strings:
``42'' not 42, ``1.0'' not 1.0, and ``x'' not 'x'. 

String arguments can be set to any string, or NULL. 

Unlike boolean options which can be ``on'' or ``off'', or string
options that can be set to a string or NULL, there is no way to set an
integer, real, or char argument off. These options must always have a
valid value. This prevents one from using the same option to turn on a
behavior and configure a parameter. For instance, say you wanted to
have \ccode{hmmbuild --evolveic 1.0} mean ``turn on the evolveic mode,
and set its free parameter to 1.0''. The getopts module encourages you
to separate this into two options, a boolean to activate the mode, and
a real to configure the free parameter; perhaps \ccode{hmmbuild
--evolveic --infotarget 1.0}.

Still, you can work around this. For a character argument, you can set
the default to the empty string \verb+""+, which will be interpreted
as a NUL character \verb+\0+, i.e.\ 0; your code can interpret value 0
as unset. For a integer or real-valued argument, you can declare the
argument as an untyped \ccode{eslARG\_STRING} with a default value of
NULL, then have your code interpret NULL as ``off''.  You lose type
and range checking when you do this, of course, and you have to
convert the string to the appropriate type yourself.
   
There is no way to turn a boolean option off by a command line option,
environment variable, or configuration file. Booleans (and strings,
for that matter) can only be turned on when their option is
selected. Booleans can be set to off by default, or toggled off
indirectly by another option is turned on (see the section on toggle
groups further below).

   \subsubsection{Connecting an option to an environment variable}

When a non-NULL environment variable name is connected to an option,
\ccode{esl\_opt\_ProcessEnvironment()} will look for that name in the
environment and sets the option value accordingly. Boolean options are
set by setting the environment variable with no argument, for instance
(in a C-shell),

\begin{cchunk}
  % setenv FOO_DEBUGGING
\end{cchunk}

and other options are set by setting the envvar to the appropriate
argument, for instance (in a C-shell),

\begin{cchunk}
  % setenv FOO_DEBUG_LEVEL 5
\end{cchunk}

   \subsubsection{Range checking}

If a non-NULL range is provided, a configured argument (including the
specified default setting) will be checked to be sure it satisfies a
lower bound, upper bound, or both. Range checking only applies to
integer, real, and char arguments. Boolean and string arguments should
set their range fields to NULL.

In a range string, a character \ccode{n}, \ccode{x}, or \ccode{c} is
used to represent the argument. Bounds may either be exclusive ($<$ or
$>$) or inclusive ($>=$ or $<=$). Examples of range strings specifying
lower bounds are \ccode{"n>=0"}, \ccode{"x>1.0"}, and
\ccode{"c>=A"}. Examples of range strings specifying upper bounds are
\ccode{"n<0"}, \ccode{"x<=100.0"}, and \ccode{"c<=Z"}. Examples of
range strings specifying both lower and upper bounds are
\ccode{"0<n<=100"}, \ccode{"0<=x<=1"}, and \ccode{"a<=c<=z"}.

Range checking occurs before any option is set.

   \subsubsection{Setting toggle groups of options}

If a non-NULL string \ccode{toggle\_opts} of ``toggle-tied'' options is
set for option X, this is a comma-delimited list of additional options
that are turned off when option X is turned on. This allows the
application to define a set of options for which only one may be
on. The application would set an appropriate one to be on by default,
and the others to be off by default.

For example, if you configure an option \ccode{-a} to have a
\ccode{toggle\_opts} of \ccode{"-b,-c"}, then whenever \ccode{-a} is
turned on, both \ccode{-b} and \ccode{-c} are automatically turned
off. You'd also want to set \ccode{toggle\_opts} for \ccode{-b} to be
\ccode{"-a,-c"} and \ccode{toggle\_opts} for \ccode{-c} to be
\ccode{"-a,-b"}.

Although booleans may only be turned ON when their option is present,
you can easily get the semantics of an on/off switch by defining
another option that works as the off switch when it is selected. For
example, you could define (GNU-ish) boolean options \ccode{--foo} and
\ccode{--no-foo}, and set \ccode{toggle\_opts} for \ccode{--foo} to be
\ccode{"--no-foo"} and vice versa.  

Toggle-tying should only be used for boolean options, but it will also
work for string options (where turning a string option off means
setting it to NULL). Toggle-tying an integer, real-valued, or char
option will result in undefined behavior, because these options may
not be turned off.

Toggling behavior occurs immediately, whenever an option with a
non-NULL \ccode{toggle\_opts} field is set.

   \subsubsection{Specifying required or incompatible options}

If a non-NULL string \ccode{required\_opts} is provided for option X,
this specifies a comma-delimited list of additional options that must
be on if option X is set. 

One case where this behavior is useful is when one (primary) option
turns on a mode of application behavior, and other (secondary) options
configure that mode. If a user tried to set the secondary options
without turning on the mode in the first place, the application should
issue a warning. So, if a mode was turned on by \ccode{--foomode} and
configured by \ccode{--foolevel <x>}, one could set
\ccode{required\_opts} to \ccode{"--foomode"} for the option
\ccode{--foolevel}.

Required options are validated when the application calls
\ccode{esl\_opt\_VerifyConfig()}, presumably after all configuration
information has been processed. This delayed verification allows the
primary options to be set anywhere and in any order, before or after
secondary options are set.

The \ccode{incompat\_opts} field is the converse of
\ccode{required\_opts}.It specifies a comma-delimited list of options
that may \emph{not} also be on if option X is on.

   \subsubsection{Example of a more fully featured \ccode{ESL\_OPTIONS} array}

The test driver in \ccode{getopts.c} uses an options array that
exercises all the optional features at least once:

\input{cexcerpts/getopts_bigarray}


\subsection{Command line parsing and beyond: config files and the environment}

Once a \ccode{ESL\_GETOPTS} object has been loaded with an options
array and initialized to default state by
\ccode{esl\_getopts\_Create()}, a \ccode{esl\_opt\_ProcessCmdline()}
call then processes all the options on the command line, updating the
configuration. 

Internally, the object keeps track of where the options end and
command line arguments begin. The macro \ccode{esl\_opt\_ArgNumber()}
returns the number of arguments remaining, which an application can
use to verify that the command line contains the expected number of
non-option arguments.  Subsequent calls to
\ccode{esl\_opt\_GetCmdlineArg()} recover the arguments in order, and
these arguments may optionally be type and range checked.

The getopts module can configure options not only via the command
line, but via environment and/or config files.  Connections to the
environment -- the \ccode{env\_var} field of the options array -- are
processed by a \ccode{esl\_opt\_ProcessEnvironment()} call.  An open
config file is processed by a \ccode{esl\_opt\_ProcessConfigfile()}
call. (The format of a config file is described below.) The
application may process any number of config files -- for instance,
there may be a master configuration installed in a system directory,
and a personalized configuration in a user's home directory.

The order of the different \ccode{Process*()} calls defines the
precedence of who overrides who. For example, in the following code
fragment:

\begin{cchunk}
   ESL_GETOPTS *g;        /* a created, initialized getopts config  */
   FILE *masterfp;        /* a master config file, open for reading */
   FILE *userfp;          /* a user's config file, open for reading */

   esl_opt_ProcessConfigfile(g, "/usr/share/myapp/master.cfg", masterfp);
   esl_opt_ProcessConfigfile(g, "~/.myapp.cfg",                userfp);
   esl_opt_ProcessEnvironment(g);
   esl_opt_ProcessCmdline(g, argc, argv);
\end{cchunk}

the precedence is defined as: defaults, master config file, local
config file, environment, command line arguments. 


\subsection{Configuring an application that uses getopts}

(This section might usefully by cut and pasted into the documentation
for a specific application, with modifications as appropriate.)

   \subsubsection{Command line option syntax}

Command line option syntax is standard, essentially identical to the
syntax used by GNU programs.

Options are either short or long. Short options are a single character
preceded by a single \ccode{-}; for example, \ccode{-a}. Long options
are preceded by two dashes, and can have any wordlength; for example,
\ccode{--option1}.

If a short option takes an argument, the argument may either be
attached (immediately follows the option character) or unattached (a
space between the optchar and the argument. For example, \ccode{-n5}
and \ccode{-n 5} both specify an argument \ccode{5} to option
\ccode{-n}.

Short options can be concatenated into a string of characters;
\ccode{-abc} is equivalent to \ccode{-a -b -c}. (Concatenation may
only be used on the command line, not in configuration files or in
fields of the \ccode{ESL\_OPTIONS} structure array.) Only the last
option in such a string can take an argument, and the other options in
the optstring must be simple on/off booleans. For example, if
\ccode{-a} and \ccode{-b} are boolean switches, and \ccode{-W} takes a
\ccode{<string>} argument, either \ccode{-abW foo} or \ccode{-abWfoo}
is correct, but \ccode{-aWb foo} is not.

For a long option that takes an argument, the argument can be provided
either by \ccode{--foo arg} or \ccode{--foo=arg}.

Long options may be abbreviated, if the abbreviation is unambiguous;
for instance, \ccode{--foo} or \ccode{--foob} suffice to active an
option \ccode{--foobar}. (Like concatenation of short options,
abbreviation of long options is a shorthand that may only be used on
the command line.)

Long option names should contain only alphanumeric characters,
\ccode{-}, or \ccode{\_}. They may not contain \ccode{=} or \ccode{,}
characters, which would confuse the option argument parsers. Other
characters should work but are not recommended.

Multi-word arguments may be quoted: for example, \ccode{--hostname "my
host"} or \ccode{-s "my string"}. However, only the space-separated
syntaxes work for quoted multi-word arguments. \ccode{--hostname="my
file"} and \ccode{-s\"my string\"} both fail.

   \subsubsection{Configuration file format}

Each line of a configuration file contains an option and an argument
(if the option takes an argument). Blank lines are ignored.  Anything
following a \ccode{\#} character on a line is a comment and is
ignored. The syntax of options and arguments is stricter than on
command lines.  Concatenation of short options is not allowed,
abbreviation of long options is not allowed, and arguments must always
be separated from options by whitespace. For example:

\begin{cchunk}
   # Customized configuration file for my application.
   #
   -a             # Turn -a on.
   -b             # Turn -b on.
   -W arg         # Set -W to "arg"
\end{cchunk}





